\documentclass[a4paper,12pt]{article}
\usepackage{geometry}
\usepackage{theapa}
\geometry{top=1.7in}

%opening
\title{\textbf{Knowing What I \emph{Mean} To Do}}
\author{Derek Lam}
\begin{document}

\maketitle

\begin{abstract}
Philosophers have tried to find ways to account for Anscombe's observations that our practical knowledge is (i) non-observational and (ii) not just about our own intentions but also external objects. These observations lead to what John Schwenkler calls the Self-Knowledge Problem. Although practical knowledge isn't just knowledge about our own intentions, according to (ii), it's common to think that our privileged access to our own intentions is the key to understanding the non-observational nature of practical knowledge. I'll present Sarah Paul's Inferential Theory as a recent attempt to develop this approach. Then, I'll argue that such an approach to practical knowledge is plausible only insofar as it's accompanied by a certain kind of theory about our privileged access to our own intentions. And I'll show that the prospect of finding the right kind of theory of privileged access to match the Inferential Theory isn't promising. On that basis, I'll suggest that the non-observational character of practical knowledge shouldn't be explained in terms of our privileged access to our mind at all. Instead, drawing inspiration from the Kripkean contingent a priori, I'll present and defend what I'll call the Stipulative Meaning Theory about practical knowledge.
\end{abstract}

\section*{1 Anscombe on Practical Knowledge}
When a person intentionally does something, he knows what he's doing. An agent's knowledge of what he does intentionally is what G.E.M. Anscombe calls ``practical knowledge''.\footnotemark Practical knowledge is said to cause the intentional actions (hence being practical) in the sense that an agent's practical knowledge is constitutive of the very intentional action he performs (see Moran 2004: 47). A person who doesn't know that he's doing X isn't doing X intentionally. Understanding the nature of practical knowledge is an important part of the investigation of intentional action.

\footnotetext{The term ``practical knowledge" is sometimes used to refer to the knowledge of how to do certain things. In this essay, I work largely within an Anscombian framework, where ``practical knowledge" just refers to an agent's knowledge about what he/she is doing intentionally. This is merely a matter of terminology. And this terminological choice does not rule out that there might be an intimate link between the two kinds of knowledge as Setiya (2008) argues.}
	
In \emph{Intention}, Anscombe claims that practical knowledge is non-observational: ``Intentional actions are a sub-class of the events in a man's history which are known to him not just because he observes them." (2000: 22)\footnotemark Let's call this the \emph{Non-Observational Thesis}. Here's Anscombe's famous example for illustrating the idea:

\footnotetext{Notice that Anscombe is making it quite clear here that, by the adjective ``non-observational", all she means is that we do not know what we do intentionally via observing what we do. This claim does not exclude that some observations have to be in in the background for practical knowledge to happen. For this reason Velleman thinks that the term ``non-observational" is a bit misleading (see Velleman 1989: 18-20). I do not share the sentiment that the term is misleading. After all, the same is true of the analytic/conceptual truths. My knowledge that bachelors are unmarried is non-observational. But of course I need to have empirical knowledge about humans and the social institution of marriage in the background to be in the position to understand and know the analytic truth. The fact that we need observational information to play an enabling role doesn't mean the observations do any justificatory work.}

\begin{quote}
A very clear and interesting case of this is that in which I shut my eyes and write something. I can say what I am writing. And what I say I am writing will almost always in fact appear on the paper. Now here it is clear that my capacity to say what is written is not derived from any observation. In practice of course what I write will very likely not go on being very legible if I don't use my eyes; but isn't the role of all our observation-knowledge in knowing what we are doing like the role of the eyes in producing successful writing? That is to say, once given that we have knowledge or opinion about the matter in which we perform intentional actions, our observation is merely an aid, as the eyes are an aid in writing. Someone without eyes may go on writing with a pen that has no more ink in it; or may not realize he is going over the edge of the paper on to the table or overwriting lines already written; here is where the eyes are useful; but the essential thing he does, namely to write such-and-such, is done without the eyes. So without the eyes he knows what he writes; but the eyes help to assure him that what he writes actually gets legibly written. (ibid: 53)
\end{quote}

I intentionally write such and such on the board. I know that I'm writing such and such on the board as the action is being performed. According to Anscombe's view, my visual experience assists the execution of my intentional action in the same way that my memory of the spellings of words helps me write on the board. I don't learn that I'm writing such and such on the board because of my memory. Similarly, I don't learn that I'm writing such and such on the board because of my visual experience. The senses (just like my memory) are practical aids for my execution, not epistemic access to what I'm doing.

Although this helps us get a grip on the intuitive appeal of the negative claim that practical knowledge is not observational, we're still left without a clue how to understand the positive phenomenon that we can know what we're doing non-observationally. We can't be content simply to show how observations seem irrelevant in epistemic access. To properly defend the Non-Observational Thesis, we need to be able to articulate how exactly then we learn about what we're doing. This is the main goal of this essay.

\section*{2 The Self-Knowledge Problem}
One might be tempted to defend Anscombe's Non-Observational Thesis by appealing to the view that intentional action just is an act of the will, something ``internal". When I intentionally write on the board, the intentional action is nothing more than an act of the will; everything else --- the picking up of a piece of chalk, the movements of my arm, the chalk marks being left on the board, etc. --- is strictly speaking consequences of the intentional action, not part of the intentional action itself. One might think that, given that intentional actions are purely mental and internal, of course the agent doesn't have to look around to know what he's doing.

The Non-Observational Thesis cannot be defended this way. As Anscombe points out, it's problematic to push intentional actions back ``first to the bodily movement, then perhaps to the contraction of the muscles, then to the attempt to do the thing, which comes right at the beginning" into the internal sphere (ibid: 53). An intentional action has to be intentional under a proper description, and the agent's practical knowledge about his own action has to be under that description. She argues:

\begin{quote}
The only description that I clearly know of what I am doing may be of something that is at a distance from me. It is not the case that I clearly know the movements I make, and the intention is just a result which I calculate and hope will follow on these movements. (ibid: 53)
\end{quote}

\begin{quote}
I am able to give a much more exact account of what I am doing at a distance than of what my arm is doing. [\textellipsis] In general, as Aristotle says, one does not deliberate about an acquired skill; the description of what one is doing, which one completely understands, is at a distance from the details of one's movements, which one does not consider at all. (ibid: 54)
\end{quote}

We know what we do as writing on the board, baking bread, walking one's dog, etc., not just some internal flickers of the will. Intentional actions often consist of happenings beyond our skull.

Pushing the intentional actions back into our skull isn't a way to interpret and defend the Non-Observational Thesis. But this is exactly what makes our non-observational practical knowledge puzzling: if our intentional actions can consist of happenings beyond our skull, knowledge about our intentional actions may consist of knowledge about external happenings. Knowledge isn't simply true beliefs; the beliefs should be justified too. How can our beliefs about external happenings be justified non-observationally? This is a way to flesh out what Schwenkler calls the Self-Knowledge Problem:

\begin{quote}
In each case, the prima facie difficulty with treating an agent's knowledge of such actions as non-observational arises from the fact that (1) it seems impossible actually to engage (and not merely: try to engage) in an action of the type in question without effecting changes in the world that involve more than the movements of one's body, but (2) it appears that an agent will often know only by sense perception that he is bringing such extra-bodily changes about, and so (3) as a consequence, it seems that in cases like these, an agent will often know only by sense perception what he is intentionally doing. [\textellipsis] Call the problem of explaining how, in cases like these, agents can know without observation whatever they are intentionally doing the self-knowledge problem for the theory of intentional action. (2011: 139)
\end{quote}

The Self-Knowledge Problem presents an important desideratum our positive account of the Non-Observational Thesis needs to meet. Our account must make sense of the idea that learning about what we're doing involves learning about external happenings non-observationally.

Even if we follow Anscombe and refuse to push intentional actions back into our head, it's still tempting to think that the non-observational nature of our practical knowledge ultimately has something to do with our privileged access to our own intentions. In the following, I'll present Sarah Paul's Inferential Theory as a recent attempt to pursue this tempting option \textbf{(section 3)}. By defending Paul's view against Schwenkler's objection \textbf{(section 4)}, I'll show that, as an account of the Non-Observational Thesis, the Inferential Theory must be accompanied by a certain kind of theory about our privileged access to our own intentions \textbf{(section 5)}. I'll argue that the prospect of finding the right kind of theory of privileged access isn't promising \textbf{(section 6)}. On that basis, I'll suggest that the Non-Observational Thesis shouldn't be explained in terms of our privileged access to our mind at all. Instead, I'll present and defend what I'll call the Stipulative Meaning Theory \textbf{(section 7 - 9)}.

\section*{3 The Inferential Theory}
Paul sums up her Inferential Theory about practical knowledge in the following passage:

\begin{quote} The central idea behind the Inferential Theory is that our beliefs about our intentional actions are not spontaneous, but evidence-based --- based largely, I will argue, on our knowledge of what we intend to be doing, where intentions are not themselves beliefs. (2009: 9) \end{quote}

The basic structure of the proposal is quasi-Gricean in that what the agent knows non-evidentially is what he intends to be doing, while insofar as he has a belief about what he is actually doing, it is evidentially based on his knowledge of what he intends, plus his evidence for thinking he is doing what he intends. (ibid: 12)

According to Paul, an agent who is doing X intentionally has the belief that he's doing X. And his belief is justified inferentially based on information about: (i) his intention for doing X, (ii) his ability to do X, (iii) his circumstance is conducive to his doing X, etc. (2009: 15)\footnote{Paul refrains from giving us an exhaustive list of the parameters that would factor in the inference. She only listed these three as examples.} A similar approach to practical knowledge is also pursued by Newstead (2006), who argues that we have introspective knowledge of our intentions, which provide defeasible evidence for what we are doing (and will be doing).\footnotemark

\footnotetext{Though, by emphasizing the fallibility of this evidential support for our beliefs about our own actions, I believe Newstead has misconstrued the main push back against the approach. She thinks that people resist the theory because one can have intention to X without doing X (hence, implicitly assuming that intentions cannot be evidence for intentional actions because of fallibility). She articulates what she believes to be the major push back against the approach in the following way: \begin{quote} [T]here appears to be an epistemic gap between having an intention and performing a certain intentional action. In each case, we have no difficulty with the idea that agents know their intentions without observation (by introspection), but the conclusion falls short of [\textellipsis] the claim that agents know their intentional actions without observation. (2006: 195) \end{quote} If this were the major push back, then Newstead's appeal to fallibilism is exactly the right response. But I have to say I don't see any textual evidence for thinking that that is her opponents' major complaint, nor do I see any reason for thinking that anyone ever make such an assumption.}

How does the Inferential Theory explain the Non-Observational Thesis without running into the Self-Knowledge Problem? Whereas information about (i) is obtained non-observationally by the agent via his privileged access to his own intentions, the same cannot be said about the agent's information about (ii) and (iii), which has to be learned empirically. So, the inferential ground for our practical knowledge consists partly of information that we learn by observation. According to Paul, this doesn't contradict the Non-Observational Thesis because the empirical information about (ii) and (iii) isn't directly about the intentional action: ``the experience and observation on which the agent's belief is partly based is not experience or observation of the particular action in question" (2009: 16). In other words, instead of doing the justificatory work for my belief about the intentional action I'm performing, the empirical information plays only an \emph{enabling role}, allowing me to draw the inference about my intentional action based on what I learn non-observationally about my own intentions. There is no reason to doubt that we can make inferences about external happenings based on non-observational information about ourselves. E.g., I can infer that my pillow is too soft based on my neck discomfort. Therefore, we have a theory of non-observational practical knowledge that is partly about external happenings.

\section*{4 Non-Observational As Privileged}
Schwenkler isn't satisfied by Paul's theory as an answer to the Self-Knowledge Problem:

\begin{quote} Is Paul's inferential view a viable response to the self-knowledge problem? One question is whether Paul fully preserves the first-personal/third-personal asymmetry that was emphasized at the beginning of this paper [\textellipsis]. For on an account like Paul's while there will clearly be fundamental differences between your and my respective ways of knowing about my intentions, it appears that once that knowledge is in place, you may be in every bit as good a position as I am to know what I am doing on the basis of the knowledge of those intentions plus my general tendencies and the evident favorability of my circumstances: this is an inference that you are every bit as capable of making as I am, and it seems that you will be no less justified in it. But it seems as if it is knowledge of one's actions, and not just of the intentions that underlie them, that ought to be characterized by first-personal privilege, and it is a significant defect of Paul's position if it fails to ensure this. (2011: 143) \end{quote}

The objection basically boils down to this. You and I have the same power of inference. Assume the Inferential Theory is true. Although you and I have different kinds of data to begin with (e.g., my access to my intentions is non-observational and your access to my intentions is not), the way I draw inferences about my own intentional actions and the way you draw inferences about my intentional actions are essentially the same. Given the same data, you would be able to draw the same inferences as I do. Hence, the Inferential Theory about practical knowledge fails to explain our practical knowledge's non-observational character as a feature that is unique to the agents' first-personal perspective.

This isn't a good objection. It rests on the misunderstanding that the point of the Inferential Theory is to explain the non-observational nature of our practical knowledge by appealing to the fact that it is \emph{inferred instead of observed}. If that were the point of the theory, then it would be fair to point out that other people's third-personal knowledge about our actions is also inferred. But that isn't the point of the theory.

According to the Inferential Theory, the key difference between my practical knowledge about my own intentional actions and your theoretical knowledge about my intentional actions is this: whereas my inference is ultimately based on my non-observational knowledge about my own intentions, you don't have such privileged access to my intentions to serve as the basis for your inference.

What Schwenkler misses, I fear, is that the Inferential Theory has two components --- (a) privileged access to intention and (b) inference from mind to world, although only the latter is emphasized in answering the Self-Knowledge Problem. Our privileged access to our own intentions explains what makes our inferred practical knowledge non-observational in its unique way. The notion of inference is not supposed to explain the uniquely non-observational nature of practical knowledge, but to explain how we can get knowledge about external happenings simply by looking inward, so to speak.

\section*{5 The Wrong Kind Of Privilege}
As a proper solution to the Self-Knowledge Problem, the Inferential Theory relies heavily on the idea that we have a privileged access to our own intention, which seems to be a relatively repressed aspect of the theory. Taking this aspect for granted when we deal with the Self-Knowledge Problem is a mistake. The plausibility of answering the problem in terms of mind-to-world inferences ultimately relies on whether there is a theory about said privileged access that suitably sheds light on the uniquely non-observational way we know about our own actions. There are different ways to understand our privileged access to our own intentions. A straightforward way would be to say that we can introspect our own intentions without relying on observation.

But Paul doesn't think that this is a good way to understand our privileged access to our intentions:

\begin{quote} [I]ntrospection alone cannot account for [our first-personal authoritative access to our own attitudes] for a model on which we are limited to discovering our own attitudes cannot say what is wrong with finding out that one has an attitude one sees no reason to have. A subject who learns through perceptual means that he has a belief or an intention for which he sees insufficient justification will not be in a position to report on this attitude in a first-personally authoritative way, for it will not seem to him to be an attitude he himself ought to have. (2012: 330) \end{quote}

The argument relies heavily on Moran's (2001) view on the first-person. Moran argues that the privileged and authoritative first-person awareness of our own mental states is primarily about the fact that our own mental states can and often present themselves to us in a way that, in virtue of being our mental states, imposes practical demands upon us:

\begin{quote} [T]he person's consciousness of his activity is not something that stands outside it observing but infuses and informs it, making a describable difference in the kind of activity it is. (31) \end{quote}

\begin{quote} The special features of first-person awareness cannot be understood by thinking of it purely in terms of epistemic access [\textellipsis]. Rather, we must think of it in terms of the special responsibilities the person has in virtue of the mental life in question being his own. [\textellipsis] [S]elf-consciousness makes a difference to what the person's responsibilities and capacities are, with respect to his involvement in their development (32) \end{quote}

Whereas Moran talks about our mental states' presenting themselves to us by imposing upon us certain responsibilit-\emph{ies}, Paul highlights one specific responsibility that our mental states present us in the quote: that our mental states present themselves as rational, i.e., as states that we should be in.

Paul thinks that, if our access to our attitudes is detached from our seeing those attitudes' reasonableness, learning about our mental attitudes won't inform what we should do with it. In particular, if I learn that I have the attitude X but have no idea whether X is reasonable, my knowledge about my attitude X wouldn't be able to tell me whether I should say that I have the attitude X when asked. Since introspection cannot inform us about the reasonableness of our mental states, the authority and privilege we enjoy in reporting our own attitudes, including intentions, cannot be explained by introspection.

I'm not entirely convinced that being able to pick up an attitude's reasonableness is a pre-requisite for letting it guide our activities and explaining our first-person authority about it. There are a lot of mental states that guide our activities without our having a sense that we ought to have those mental states, for example, when the mental states I'm in when I'm daydreaming. We also seem to report those mental states with first-personal authority all right. So, I don't find Moran's and Paul's reason against using introspection to explain our privileged access (to our own intentions) particularly compelling.

\emph{That being said}, I agree with their conclusion that introspection isn't the right place to look for an account of our privileged access to our intentions --- \emph{\textbf{if}} the goal is to plug the account into the Inferential Theory to produce a proper answer to the Self-Knowledge Problem. Notice that practical knowledge is meant to be non-observational in a special way. It's non-observational in virtue of \emph{the agent's active role} in performing the relevant intentional actions. An account of the Non-Observational Thesis ought to shed light on this connection. Introspection, usually conceived as an inward-looking gaze, is at best a special way in which information about our mental states is given to us. Such a passive source is poorly designed for the job. Even if I accept that we have introspective access to our intentions, it cannot be in virtue of this that our practical knowledge is non-observational in its unique way. The same objection, I believe, applies to any attempt to understand our privileged access to our own intention as some form of inner-sense.

(It's instructive to notice that what I have just said isn't an objection to understanding our privileged access via introspection or inner-sense; it's only an objection to theories that explain our non-observational access to our intentional actions via our introspection or inner-sense to our own intentions. I have no complaint about understanding our privileged access via introspection as long as one explains our access to our actions by something else.)

The importance of accounting for the Non-Observational Thesis by connecting it to the agent's active role has been rightly highlighted by Schwenkler:

\begin{quote} Well, what exactly do we mean in this context when we speak of an observer, and of knowledge had by observation? Surely, the implication of a reliance on sense perception is important here, but the notion of observation connotes more than this: observation involves not just perceiving something, but doing so somehow passively; to observe something is to sit back, as it were, and simply take it in for whatever it happens to be.

[\textellipsis] On my reading, O'Shaughnessy's point in denying that a person can ever observe his intentional action is not that human agents are never sense-perceptually aware of the things they intentionally do [\textellipsis], but rather that the kind of self-awareness characteristic of intentional agency is essentially bound up in the activity of self-control. [\textellipsis] [T]he agent is a ``creator", and his awareness of his actions is always subordinated to the task of ensuring that things proceed along the course he intends. (2011: 146-147) \end{quote}

Schwenkler is right in pointing out the need for a connection. But I'm not convinced by the particular way he then goes on to establish the connection. On my reading, what does the heavy lifting in Schwenkler's view is the fact that he defines the notion of ``non-observational" to mean being more than a piece of observation (2011: 147-148). He then solves the Self-Knowledge Problem by first embracing the claim that practical knowledge is indeed justified observationally and then pointing out that our practical knowledge also serves the purpose of guiding our intentional actions. Our practical knowledge is not just observational data that sit there; it plays a role in shaping the way we act. Therefore, our practical knowledge can be said to be more than a piece of observation, i.e., non-observational in Schwenkler's sense, despite being observationally justified (see also Schwenkler 2015).

Schwenkler's account appears to have changed the subject. I don't think the Self-Knowledge Problem should be solved by conveniently redefining ``non-observational". If Anscombe's Non-Observational Thesis is philosophically intriguing, that's partly because it seems to be getting onto something unique and puzzling about the way agents' epistemic access to their own intentional actions, not just about how the agents use their knowledge about their actions. By making the Non-Observational Thesis not about the epistemic access but the way agents use their practical knowledge, Schwenkler isn't shedding light on the phenomenon but talking about something else.

Our knowledge about our own action indeed helps shape our action --- hence being practical; it's also special in the sense that we don't need observation to acquire it despite the fact that it can be partly about external happenings --- hence being non-observational. These are two intriguing phenomena about our actions that we need to address. Schwenkler's approach collapses the two phenomena into one. I think it's a fair option to \textbf{\emph{reject}} the Non-Observational Thesis (e.g., see Gr\"unbaum 2013). And I find it more appropriate to describe Schwenkler's view as going in that direction instead of as an attempt to interpret the Non-Observational Thesis.\footnotemark I happen to think that Anscombe is on to something by the thesis and I'm here focusing on the more modest task of trying to figure out, if Anscombe is right that we have non-observational epistemic access to our own intentional actions, how we should make sense of such an epistemic access in the face of the Self-Knowledge Problem. It's for this pragmatic reason that I'll set Schwenkler's answer to the problem aside.

\footnotetext{Schwenkler is more candid about this in a later paper: ``I'll argue, \emph{contra} Anscombe and several of her recent interpreters, that the `materiality' of human action [\textellipsis] means that our knowledge of what we are doing can be partly receptive, and so grounded in our sensory perception of our movements and their effects." (2015: 18)}

Paul doesn't explain our privileged access in terms of introspection/inner-sense. Instead, she draws inspiration from Moran's authorship approach by appealing to the fact that ``the mental act of deciding to $\varphi$ is a paradigmatic way of entering into the state of intending to $\varphi$". (2012: 336) The idea is that, since making conscious decisions is the paradigmatic way we \emph{produce} our intentions, one can safely \emph{infer} that one has the intention to $\varphi$ based on the fact that one has made a conscious decision to $\varphi$. Let's call this the \emph{Decision View}.

Making a conscious decision is a mental act. Unlike the appeal to introspection, plugging the Decision View about our privileged access to intentions into the Inferential Theory of practical knowledge gives the agent's activeness a role to play in the epistemic story. By actively making a conscious decision, I'm non-observationally aware of the decision being made and, due to the fact that decisions very reliably produce intentions, inferentially justified to form the belief that I intend to do whatever it is that I have decided to do. With this non-observational knowledge about my own intention, I'm then inferentially justified to believe that I'm doing whatever it is that I intend to be doing. Thus, I have practical knowledge that is inferentially based on the non-observational ground that I have made a conscious decision to act.

\section*{6 Activities Remembered}

The Inferential Theory plus introspection isn't an answer to the Self-Knowledge Problem because the agent's activeness plays no role in the resultant picture of practical knowledge. Following Paul, we replace introspection with inference based on conscious decision. Does the Inferential Theory plus Decision View combo offers a good account for the Non-Observational Thesis and a good answer to the Self-Knowledge Problem? I think it doesn't. Two objections are particularly pressing given our goal is to understand practical knowledge's uniquely non-observational nature.

Let's begin with two observations. First of all, notice that, if Paul is right that, by making conscious decisions, we produce intentions, then the conscious decisions and the intentions they produce happen at different moments: one makes a decision; then comes the corresponding intention. Secondly, practical knowledge, as long as it's knowledge, requires justification.\footnote{Some philosophers deny this. Velleman and Setiya, for example, think that practical knowledge is a special kind of knowledge that doesn't require justification. They think that that an agent is exercising her know-how is sufficient to secure agential knowledge. I disagree. Note that exercising know-how surely secures safety for the agent's beliefs about her own actions. But securing safety is just to eliminate a factor that can undermine a belief's status as knowledge. That isn't telling us anything \emph{positive} that makes a mere belief knowledge. Justification is probably not enough, but required.} Notice that the justification and the justified belief must be simultaneous. Having a piece of evidence only at t but not at a later time t' doesn't provide justification for a belief at t'.

Based on the above two observations, Paul's view actually implies that my awareness in making a conscious decision doesn't justify the belief about the presence of the intention produced by the decision. Instead, it's my \emph{\textbf{memory}} that a conscious decision was made that gives me inferential justification for believing that I have the relevant intention. Since the temporal distance between decision and intention is inevitable, it's inevitable --- by the light of her view --- that our practical knowledge is inferentially based on our memory about our conscious decisions.

Memory justifies regardless of whether the memory is veridical. The fact that our conscious decisions can only appear in Paul's picture of practical knowledge as something that features in the content of our memory implies that our active decisions aren't actually the things doing the justificatory work, which is done by our memory of our decision-makings. It's an illusion that the agent's activeness itself plays any real epistemic role in the picture of practical knowledge Paul offers. Although practical knowledge is indeed non-observationally justified according to the view, the non-observational character isn't in any way rooted in the agents' activeness in the relevant intentional actions. So, for the purpose of making sense of the non-observational character of our practical knowledge, her view is no better than Inferential Theory plus introspection.\footnote{This objection doesn't rely on internalism about justification.}

The essential role memory plays in Paul's view leads to a second objection. The Non-Observational Thesis is meant to capture something unique about our practical knowledge, which is about my \emph{current} actions. (Unlike my knowledge about my current actions, my knowledge about my future and past actions isn't constitutive of those actions; such knowledge is therefore not practical.) I know that I'm currently painting the wall. And I know that I painted the wall ten years ago. Intuitively, the first piece of knowledge is non-observational in a special way that the second piece of knowledge isn't. There is supposed to be a qualitative difference. Yet, under Paul's view, the two pieces of knowledge collapse into the same kind. In both cases, I have the memory of making a conscious decision, based on which I have the inferentially justified belief that I'm painting the wall and the inferentially justified belief that I painted the wall ten years ago. The two pieces of knowledge may differ in certainty, but not qualitatively. By plugging the Decision View in the Inferential Theory, we lose the phenomenon we want to make sense of.

Here's a quick recap. It's tempting to think that the explanation for Anscombe's Non-Observational Thesis lies in our privileged access to our intentions. In the form of Paul's Inferential Theory, this tempting idea is developed in a way that avoids the Self-Knowledge Problem. The success of this approach requires a theory about our privileged access to our intentions that gives our activeness as agents an integral role in explaining our practical knowledge's uniquely non-observational character. Typically, this involves either appealing to our introspection/inner-sense or appealing to our authorship of our own intentions. Based on what we have seen, the prospect seems dim. I consider this a good reason for exploring the possibility that the root of our practical knowledge's non-observational character may not rest in our privileged access to our intentions after all.

(One might protest that I have left out Velleman's (1989) and Setiya's (2008; 2009) view, which is often presented as offering a theory of practical knowledge to solve the Self-Knowledge Problem. But I think that's a misrepresentation of their view. On my reading, their theory is about explaining our justification for \emph{forming} a belief that we are performing certain actions \emph{before} we have that belief. The Self-Knowledge Problem, on the contrary, is about what justification we have for the belief that we are performing certain actions \emph{as we are entertaining that belief and performing the actions}. Those are different issues. Defending my reading of Velleman and Setiya requires detail analysis of their writings that has to be left for another occasion. But in any case, I think Paul (2009) has offered a forceful response to the kind of view Velleman and Setiya propose even if it were \emph{treated as} a solution to the Self-Knowledge Problem. I don't want to reinvent the wheel here. Paul's view is, I believe, the most recent attempt that comes very close to adequately accounting for Anscombe's Non-Observational Thesis. That is why I focus on her view to illustrate the inadequacy of relying on our privileged access to our own intentions to explain the non-observational character of our practical knowledge.)\footnote{We should also resist the temptation to read Small's (2012) argument for the \emph{Cognition Condition} --- the thesis that knowing what we are doing is constitutive of intentional action --- as something that offers us any insight into the non-observational character of practical knowledge. That is because proving the Cognition Condition --- even if the proof is done from an armchair --- only tells us that intentional actions require practical knowledge; it says nothing about whether said knowledge, which is essential for intentional actions, is based on observation or not.}

\section*{7 Stipulative Meaning Theory}

We need a fresh start (\emph{alas, don't we all?}), a clean break from our privileged access to intentions. Perhaps we should look beyond what lies immediately in front of us and consider the possibility that the non-observational knowledge we are trying to make sense of is part of a broader phenomenon. Even if we set knowledge about abstract objects aside (e.g., mathematics), non-observational knowledge about the external world isn't unprecedented. Kripke (1980) famously argues that there are contingent facts about the world that one can know a priori. For example, if I stipulate that a stick at time t to be the standard unit of a new measurement scale ``schmeter", \emph{as the stipulator}, I know without having to empirically study the stick that it's 1 schmeter long at time t. This is a fact about the length of an external object I know non-observationally.\footnote{The idea of contingent a priori is itself the subject of many debates. For the purpose of this paper, I'll take Kripke's examples as sufficient for motivating the idea. See, e.g., Korman (2010) and Jeshion (2001) for further defense of the Kripkean contingent a priori.}

Suppose you ask me how I get to know that the stick is 1 schmeter long at time t. I would say, ``Of course I know, I'm the one who sets it up as the standard unit." My non-observational knowledge is not based on the mere fact that the stick at t is the standard unit. Instead, it's based on the fact that I am the one who \emph{\textbf{makes}} it the standard unit. My active role in the stipulation to make it the case that the stick is 1 schmeter long is meant to provide the rational basis for my non-observational knowledge. So, apparently, the Kripkean contingent a priori and our practical knowledge are non-observational knowledge in a similar way. That gives us a prima facie reason to expect a unified account for the non-observational nature of both our practical knowledge and my knowledge that the stick is 1 schmeter long at time t. Perhaps this can offer us the much needed theoretical anchor for thinking about the non-observational character of practical knowledge anew.

My belief that the stick is 1 schmeter is justified in virtue of the stipulative meaning of ``schmeter". Certainly, I can only know that there is a stick out there by observation. But such observational knowledge only plays an enabling role in the justification of my belief that the stick is 1 schmeter long. The justificatory work is done by the meaning of ``schmeter"; my knowledge that the stick is 1 schmeter long is, therefore, a priori. If we want a unified account of the non-observational nature of the Kripkean contingent a priori and our practical knowledge, we should say something similar about the way we know what we do intentionally.

By applying the same reasoning to our practical knowledge, we obtain what I call the Stipulative Meaning Theory about practical knowledge. Here is my proposal. By going through whatever that I'm going through in order to perform the intentional action X, I am, \emph{by my performance}, stipulating that the notion ``doing X" covers \emph{whatever happenings that I'm now involved in}. As a result, while my belief that I'm doing X is made true by those happenings that I'm now involved in (which is stipulated to be an instance of doing X by the action itself), the belief is justified \emph{\textbf{analytically}} in virtue of the stipulative meaning of ``doing X". The resulting practical knowledge is, therefore, a priori in the way my knowledge that the stick is 1 schmeter long is. (That an action can also be semantic stipulation is not a far-fetched idea at all: by handing Max a prison uniform with the number 24601 printed on it, one performs an action that stipulates that the phrase ``prisoner 24601" refers to Max.)

Just like I must rely on observation to know that there is a stick out there to begin with, I need observational knowledge of the happenings in the situation as I'm doing X. But as I have argued earlier, such observation only plays an enabling role so we can use what we observe to stipulate meaning for notions like ``1 schmeter" and ``doing X". My belief that the stick is 1 schmeter long and that I'm doing X are justified analytically due to my semantic stipulations. Hence, the knowledge is a priori in spite of the involvement of observed external objects. By the light of the Stipulative Meaning Theory, the fact that doing X consists of external objects doesn't make non-observational practical knowledge any more mysterious than a mother's knowing a priori that her newborn son is Teddy as she names him ``Teddy" herself --- even though it's an external and contingent fact that her newborn son is named Teddy. That gives us a semantic solution to the Self-Knowledge Problem.

I mentioned that this theory is a break from attempts to trace our practical knowledge's non-observational nature to our privilege access of our own intentions. It's perhaps worth emphasizing that what I mean isn't that the Stipulative Meaning Theory denies that we have such privileged access. There is no doubt that a privileged access of that sort is present when I stipulate the stick to be the standard for schmeter. For example, perhaps I must be aware in a privileged manner that I intend to perform the stipulation. And I've no problem acknowledging that such privileged access is a non-observational epistemic source. But my knowledge that the stick is 1 schmeter long is non-observational not because it's based on our privileged access to our mind. (Note that, to do mathematics, we must probably also be introspectively aware of our reasoning process. That of course shouldn't lead us to conclude that mathematical knowledge is non-observational because it's introspection-based. Introspection can be involved in a piece of non-observational knowledge without being the \emph{explanation} of its non-observational nature.)

Of course, the success of semantic stipulation partly depends on the mercy of the environment. For example, I cannot pick a stick to be the standard of schmeter if I happen to be a BIV. (In such case, I can at best \emph{try} to do so.) Notice that \emph{performing} an action does not mean the action is \emph{successfully executed}. E.g., I might be performing the action walking to the store, but the action I'm performing may fail to be successfully executed because I'm knocked down by a car before I get to the store (e.g., see Thompson (2011)). So, it isn't easy to \emph{fail to perform} an intentional action. Still, under unfortunate enough circumstances, I could be wrong when I believe that I'm intentionally doing X. Our practical knowledge is \textbf{\emph{non-observational but fallible}}. However, my point is, \emph\textbf{{when}} an intentional action X is indeed performed, I know that I'm doing X analytically based on semantic stipulation. (It's instructive to keep in mind that I'm proposing a theory of our \emph{knowledge} about our actions, \emph{not} a theory of our actions.)

\section*{8 The Agent's Role}

The major obstacle in tackling the Self-Knowledge Problem by combining the Inferential Theory with a theory of privileged access is that the agent's activeness ends up being marginalized (even with the Decision View). Is the Stipulative Meaning Theory better? I believe so.
 
In section 6, I argued that, by locating the relevant activeness of the agent in his mental act of decision making, we create a temporal distance that alienates the agent's activeness from the non-observational justification of practical knowledge. This, I argued, misses the uniqueness of our practical knowledge's non-observational character.

According to the Stipulative Meaning Theory, the meaning of ``doing X" is in part being produced \emph{as the action is being performed}. I don't first do X by going through ABC and then in retrospect stipulate that going through ABC counts as doing X. The performance of X by going through ABC \emph{just is} the relevant act of semantic stipulation, which stipulates that going through ABC counts as doing X.\footnotemark Moreover, the semantic stipulation of a term doesn't only fixes the use/practice of a term, it's itself also a part of the use. Meaning is based on use. But the use neither causes nor temporally precedes meaning. So, the act of semantic stipulation, as part of the use, isn't meant to happen before the stipulative meaning. Instead, the stipulative meaning is in place while one performs the stipulation --- which marks (instead of causes) the beginning of the use of a term. There is no temporal distance between an act of semantic stipulation and the stipulative meaning. As a result, my belief that I'm doing X --- which I assume to be a constitutive part of my intention to be doing X --- is justified simultaneously and analytically in virtue of my performing the action.

\footnotetext{The reason for accepting the ambitious claim that an intentional act is also an act of semantic stipulation is, ultimately, based on the work it does to help us make sense of our practical knowledge properly without running into the problems that other theories face.}

By pointing at a stick and declaring ``from now on, \emph{this stick is 1 schmeter!}", the claim ``this stick is 1 schmeter" here expresses a belief that is justified analytically due to that act of declaration itself. Similarly, when I make an omelette by going around the kitchen cracking eggs and chopping chives, such external happenings are being stipulated to count as an instance of making an omelette in virtue of the mere fact that I'm intentionally making an omelette by cracking eggs and chopping chives. (Hence, my belief that I'm making omelette is analytically justified in virtue of my ongoing intentional action of making omelette itself.)

On my account, the epistemically relevant activeness of the agent is the intentional action itself --- not the prior mental act of decision. There is no temporal distance between the relevant act of the agent and the practical knowledge to be non-observationally justified. It's not the memory of something that an agent does that constitutes analytic justification for his practical knowledge. The agent's action itself does that. Hence, in the Stipulative Meaning Theory, the agent's performance plays the lead in the explanation of the non-observational character of practical knowledge. My theory therefore overcomes the insurmountable obstacle that haunts the privileged access approach.\footnote{It's worth noting that my theory of practical knowledge produces a case of what Elizabeth Barnes (forthcoming) calls symmetric ontological dependence. In the beginning of this essay, I acknowledged the Anscombian claim that the agent's knowledge of what he's doing intentionally is practical in the sense that that piece of knowledge is constitutive of the intentional action itself. Now my theory says that an intentional action is constitutive of the agent's practical knowledge because the action is constitutive of the agent's analytic justification of the belief that he's performing that action. Although philosophers are more familiar with asymmetric ontological dependence, Barnes is correct to point out that symmetric ones aren't rare. For example, a social act and a social institution often stand in a symmetrical constitutive relation. My act of spending money is part of and hence is constitutive of the financial institution; at the same time, an act of spending money is partly constituted by the financial institution that it's embedded in.}

\section*{9 Knowledge After The Act}

Since my theory purports to offer a unified account for both our practical knowledge and the Kripkean contingent a priori, what I just said applies to the latter too. This could be a source of potential worry. When I perform the act of stipulation (either just in my head or publicly), that act itself partly constitutes the beginning of the meaning of ``schmeter", according to which the stick is analytically 1 schmeter long. That is why the stipulator has a priori knowledge of the contingent fact that the stick is 1 schmeter long.

What happens \emph{after} the act of stipulation? It seems that I still know without having to observe that the stick is 1 schmeter long. But if my account says that the non-observational justification for our knowledge of our own actions and that of the Kripkean contingent a priori is constituted by our performing acts of semantic stipulation (which include all intentional actions), then that non-observational justification should cease to exist once the acts are over. That appears to contradict the fact that I still know without observation that, for example, the stick is 1 schmeter long after the act of stipulation is over.

There is actually no contradiction. It's helpful to remember that my objection to appealing to either introspection or Paul's Decision View isn't that they fail to make our practical knowledge non-observational. They can. The crux of my objection is that the kind of non-observational character those theories attribute to our practical knowledge is not the right kind of non-observational character that we are after, namely, the kind that stems from the agents' activeness in their intentional actions. Once it's clear that my view allows multiple kinds of non-observational knowledge, we should be able to see that my view only implies that, after the stipulation, the stipulator no longer has the kind of non-observational knowledge constituted by his activeness as a stipulator.

Kripke restricts his claim about contingent a priori to the stipulator:

\begin{quote} What, then, is the epistemological status of the statement ``Stick S is one meter long at t\textsubscript{0}", \textbf{for someone who has fixed the meter system} by reference to stick S? It would seem that he knows it a priori. (1980: 56; my emphasis) \end{quote}

Nonetheless, people other than the stipulator can also have a priori knowledge of the fact that the stick is 1 schmeter long. Perhaps Adam is present when I publicly stipulate that the stick is the standard unit for schmeter. As a result, he doesn't need to study and measure the stick empirically to know that it's 1 schmeter long. (Notice that the fact that Adam needs to hear my public stipulation isn't a reason for thinking that his knowledge about the schmeter stick is observational. Otherwise, my knowledge that bachelors are unmarried would be equally observational simply because I need to hear others to learn English.) Adam can also acquire knowledge of the fact that the stick is 1 schmeter long by learning via testimony that an act of stipulation happened.

It's clear, however, the stipulator's a priori knowledge about the schmeter stick is supposed to be significantly different from the way in which Adam's knowledge is a priori. The objection gets this right: according to the Stipulative Meaning Theory, once the act of stipulation is done, the stipulator's \emph{uniquely} non-observational knowledge about the schmeter stick is over. But it doesn't follow that the stipulator ceases to know that the stick is 1 schmeter long non-observationally. The stipulator's non-observational knowledge has simply turned into knowledge that is non-observational in a different and \emph{derived} manner. After the stipulation, the stipulator usually has the \emph{memory} about the fact that his former self has made the stipulation and the stipulated meaning of ``schmeter". The memory allows the post-stipulation stipulator to know non-observationally that the stick is 1 schmeter long --- as a spectator of his former self, in the same way learning about the act of stipulation allows Adam to know non-observationally that the stick is 1 schmeter long as a spectator.

Far from being a defect of my view, I think it's actually an advantage of my view that it makes available a more sophisticated understanding of the contingent a priori by bringing to light the under-appreciated distinction between a \emph{stipulator's a priori justification} and a \emph{spectator's derived a priori justification} (e.g., Adam's justification for believing that the schmeter stick is 1 schmeter long). As long as we find it plausible to say that the stipulator's knowledge is non-observational, it's hard to deny that Adam's is also non-observational. At the same time, my theory leaves room for the finer point that, although the two kinds of knowledge are both non-observational and intimately related, a stipulator's knowledge is tied to the stipulator's activeness in a way that isn't true of a spectator's derived a priori knowledge: in Adam's case, it's the awareness of my stipulation, not the agent's activeness, that does the epistemic work; and he could very easily have the same non-observational justification to believe that the stick is 1 schmeter by having misunderstood the act of stipulation --- perhaps what he witnessed was part of a play, not a genuine act of stipulation. (If that were the case, of course, Adam's non-observationally justified belief wouldn't be knowledge.)

Since practical knowledge is a kind of contingent a priori according to my theory, what I just said applies to practical knowledge. When I'm doing X intentionally, my action constitutes part of the ground for the analytic justification of my belief that I'm doing X. As I have argued earlier, that gives me a piece of non-observational practical knowledge that is intimately rooted in my activeness while doing X. This kind of non-observational knowledge expires once the intentional action is over.

But that doesn't mean we have to rely on observation to know about our past actions. As long as I have memories of the non-observational practical knowledge I had about my past actions, I have non-observational knowledge about those past actions based on my memories inferentially. However, such after-the-act non-observational knowledge is no longer a piece of knowledge based on the activeness of the agent but a piece of spectator's knowledge \emph{about my former self when I was an agent}. So, my theory implies that, whereas our knowledge about our current actions and our knowledge about our past actions can both be non-observational, they are non-observational in very different ways.

This is also a highly desirable implication. I know that I did certain things intentionally. I know that I'll do certain things intentionally. And I know that I'm doing certain things intentionally. It's possible to know about all this without going around observing things in the world. But our practical knowledge about our current actions is especially intriguing because the actions that we're currently doing appear to be transparent to us in a way that our past and future actions aren't. Insofar as this transparency (whatever it amounts to) appears to be the basis of our practical knowledge's non-observational character, it seems plausible to say that our knowledge of our current action is non-observational in its own unique way that our knowledge of our past and future actions isn't. It isn't an accident that the examples that Anscombe uses to motivate the Non-Observational Thesis are all cases of current actions.

Hence, it's actually necessary for a good account of the Non-Observational Thesis that the non-observational character of practical knowledge is explained in a way that doesn't apply to our knowledge of our past and future intentional actions. My Stipulative Meaning Theory is rather uniquely well cut out for meeting this desideratum. Not only is our knowledge of past actions not a threat to the Stipulative Meaning Theory, the way the theory handles our knowledge of past actions is actually superior to what we get from other theories.

\nocite{*}
\bibliography{Knowing_What_I_Mean_To_Do}
\bibliographystyle{theapa}


\end{document}
