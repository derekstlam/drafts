\documentclass[a4paper,12pt]{article}
\usepackage{geometry}
\usepackage{theapa}
\geometry{top=1.3in}
%opening

\title{\textbf{Modal Testimony and Facts of Modality}}
\author{Derek Lam}

\begin{document}

\maketitle

\begin{abstract}
Standing in stark contrast to the anti-individualistic trend in contemporary epistemology as a whole, contemporary modal epistemology has been very individualistic. Instead of viewing this as a flaw in modal epistemology, I gather evidence for thinking that the individualistic trend reflects a genuine phenomenon about modal justification: there is no testimonial justification for modal judgments. Then, I argue that, if we decide to take the individualistic nature of modal justification seriously, we have a good reason to accept anti-descriptivism about modal judgment.
\end{abstract}

\section*{1 Testimony about Modality}
There are many approaches to modal epistemology. There are those who based their view on conceivability or imaginability (e.g., Yablo 1993, Geirsson 2005), there are those who rest their view on postulating the faculty of modal intuition (e.g., Bealer 2002, Fiocco 2007), there are those who rely on our implicit knowledge of the principle of admissible interpretations (e.g., Peacock 2002), and there are those who appeal to our power of forming supposition (e.g., Ichikawa \& Jarvis 2012). Among these major approaches, one thing is in common: they are all individualistic. These views make it seem that we have to obtain modal justifications on our own. Conceiving, imagining, intuiting, understanding, supposing, etc. are all mental activities that we perform all by ourselves. And the logical inferences we draw based on the modal judgments justified by these mental activities don't make the whole process any less individualistic.

The individualistic tendency in the modal epistemology literature stands in stark contrast to the non-individualistic trend in the rest of epistemology as a field. The trend manifests itself in a surge of interest in the epistemology of testimony. Among other things, the idea that trust in others' testimony is an \emph{irreducible} source of epistemic justification has gained widespread attention and support among epistemologists (e.g., Hardwig 1991; Coady 1992; Lackey 1999; Foley 2001; Zagzebski 2012). The importance of non-individualistic contribution to knowledge and justification has been recognized by many epistemologists. And I'll assume this non-reductionist view about testimonial justification to be correct throughout this essay. This makes the non-social character of the current state of modal epistemology all the more intriguing.

The sharp contrast between the persistently individualistic modal epistemology and the anti-individualistic trend in epistemology prompts us to ask a question: Is the individualistic trend in modal epistemology a feature or a bug? The primary goal of this essay is to argue that it's a positive feature of modal judgments that they cannot be justified directly by testimony. I'll then explore the metaphysical implication of this feature.

\section*{2 Defending Individualistic Modal Justification}
\subsection*{2.1 ``Modal Judgment" and ``Testimony"}
Let's begin by clarifying two central notions: ``modal" and ``testimonial justification". When I speak of modal judgments, I mean judgments of possibility, necessity, or counterfactual. There are other claims that philosophers are inclined to classify as modal, e.g., claims about powers and essences. I have no objection to philosophers using the term ?modal' in a broader sense to describe those kinds of claims as well. In this essay, however, I focus on modal judgments in the narrow sense.

In this narrow sense, claims about essences are not modal claims even though they entail modal claims (just like we would not say that ``there's an apple on the table" is a modal claim just because it entails a modal claim --- ``it's \emph{possible} that there is an apple on the table"). I'm certainly not the first to use the word ``modal" in this narrow way. For example, the notion ``modal" is used narrowly when Kit Fine's account of essence is frequently described as a \emph{non-modal} account. The same applies to causal claims. They aren't modal claims, although they may entail certain modal claims.\footnotemark

\footnotetext{I admit that this clearly rules out the prospect of a counterfactual/modal \emph{\textbf{analysis}} of causation, according to which a causal claim \emph{\textbf{just is}} a counterfactual/modal claim. But as long as I am not ruling out that we can obtain counterfactual conclusions from causal claims logically, I don't consider this a big problem. There is surely a strong intuitive pull towards the idea that there is \emph{some kind of} inferential connection between causal and counterfactual judgments. But a counterfactual analysis of causation is more properly described as a theoretical aspiration or ambition. Philosophers may try to aim for it, but there is no compelling reason for us to hold ourselves committed to its eventual success. After all, it is not as if anyone has even come close to offering a necessary and sufficient condition for causation in terms of counterfactual without running into very troubling counterexamples. Furthermore, even if a necessary and sufficient counterfactual condition for causation that is free from troubling counterexamples is available someday, there is still a conceptual gap between a necessary and sufficient condition and an analysis. An analogy: Socrates exists iff singleton Socrates exists; nonetheless, the singleton's existence is not an analysis of Socrates' existence (and vice versa) --- it is not the case that the singleton's existence \emph{\textbf{just is}} Socrates' existence. By saying that causal claims are not modal in the narrow sense, I reject reduction of causation into certain modal phenomenon.}

By ``testimonial justification", I mean the irreducible, prima facie justification one gets from taking what another person says at face value. A stranger told me that ``today is going to rain". All things being equal, I'm thereby justified to take her words for it and accept that today is going to rain. What the person says matches the content of my thought. And the justificatory value of my trust in that person's testimony isn't reducible to the justificatory value of something else, e.g., the \emph{reliability} of that stranger's judgment about climate.

Suppose I'm told that ``Teddy is divorced". All things being equal, I'm justified in believing that Teddy is divorced. But then, knowing that Teddy was married to Jennet, I \emph{infer} that Jennet is divorced. According to my use of the phrase ``testimonial justification", the justification of my belief that Jennet is divorced doesn't count as testimonial justification. That's because that Jennet is divorced isn't really what I'm told.\footnotemark

\footnotetext{This distinction between genuine testimonial justification and inferential justification based on testimonial information roughly mirrors the distinction Sarah McGrath draws between the pure and impure moral deference (2011: 114-115).}

Finally, notice that a person can say something without \emph{explicitly} saying it. For example, by saying that ``I bet Simon is late again", a person is implicitly saying that Simon is always late. Therefore, there is room for implicit testimonial justification. When someone says ``I bet Simon is late again", I can acquire testimonial justification for the belief that Simon is always late. (The line between things I accept via implicit testimony and things I \emph{infer} from explicit testimony isn't sharp. But that doesn't matter for our purpose.)

\subsection*{2.2 Testimony Breaks Down}
Consider the following sentences:
\begin{quote}
\begin{description}
\item[(1)] There's an apple on the table.
\item[(2)] The Universe began with the Big Bang.
\item[(3)] Martha is my mom.
\item[(4)] Messi is injured.
\end{description}
\end{quote}

These sentences can all be justified by testimony. Take (1) for example. All things being equal, if someone tells me that there's an apple on the table, I'm justified in accepting that there's an apple on the table. The same is true for (2). I know almost nothing about the Big Bang except for the fact that it's called the Big Bang. Nonetheless, I'm justified in believing that the Universe began with the Big Bang, all things being equal. I find it rather uncontroversial that the same applies to (3) and (4).

Now consider another list of statements:
\begin{quote}
\begin{description}
\item[(5)] It's possible that there's an apple on the table.
\item[(6)] It 's possible that Derek is 5 feet taller than he actually is.
\item[(7)] It's impossible for someone other than Martha to be my mom.
\item[(8)] Barca wouldn't have won had Messi been injured.
\end{description}
\end{quote}

Let's consider (5). Suppose I look at the table and see that there is no apple on the table. I'm perceptually justified to believe that there's no apple on the table. Nonetheless, I accept that it's possible that there's an apple on the table. Suppose I'm asked to justify my modal claim that it's possible. It sounds unfitting for me to answer by saying, ``because someone told me so". It makes just as little sense for me to say, ``because I can conceive of it \emph{and also} because someone told me so". Presumably, that sounds objectionable because, when I say ``and also", I am supposed to follow up with something that adds extra justification (maybe just a little) to the claim that it's possible that there's an apple on the table. But the fact that someone else told me that it's possible doesn't seem to do that.

The same goes for (6) - (8). Take (7) as an example. It's sensible that I accept that Martha is my mother based on someone's testimony. But it isn't sensible for me to say that it's impossible for someone other than Martha to be my mother \emph{because someone told me so}. Kripke argues that we have our material origin --- and hence our parents --- essentially. I can decide whether I accept his argument after reflecting on it on my own. If I accept the argument, then I can justifiably decide for myself that someone other than Martha couldn't possibly be my mother. But one can't justify the judgment that someone else couldn't possibly be my mother based on trust.\footnotemark

\footnotetext{Now one might think that, depends on who that someone else is, this may not sound very awkward. Notice that the idea of testimonial justification is that one has prima facie justification for trusting what someone else tells us, all things being equal. And this testimonial justification is meant to be irreducible to other forms of justification. If the justification only works when we have extra information about the person, it is an indication that that justification is not a matter of trust irreducibly, i.e., it is not a matter of us taking that person's words for something. So it is not a case of testimonial justification in the narrow sense.}

Similarly, say Adam isn't a soccer fan and know nothing about soccer at all. Nonetheless, it seems perfectly reasonable for Adam to judge that Messi is injured, i.e., (4), based on the testimony of a pundit alone. We would all take Adam seriously when he tells us about it. (Perhaps it's a news that we haven't heard of.) Now say Messi isn't injured and Barcelona proceeds to win the match. And Adam tells us, ``Hey guys, it's just in. Barcelona wouldn't have won had Messi been injured. The pundit just said so." (That is, making the judgment (8) based on the pundit's testimony.) I'm strongly inclined to say Adam's claim is inappropriate. (8) isn't the \emph{kind} of judgment one can just pick up from others.

Unlike (1) - (4), testimonial justification doesn't seem to work for (5) - (8). If we compare the two lists of statements, the only noteworthy difference between them is that the first bunch of sentences are non-modal but the second bunch of sentences are modal. Somehow, testimonial justification works for ``there's an apple on the table" but breaks down when we add just a little sprinkle of modality in the judgment, i.e., ``it's possible that there's an apple on the table". This contrast gives us a good reason to think that the sprinkle of modality is \emph{responsible for the failure} of testimonial justification. Hence, we have prima facie evidence for believing that modal judgments cannot be justified by testimony.

\subsection*{2.3 Testimony Does Not Break Down (Only Apparently)}

Although there are modal judgments that seem to support my thesis that modal justification is inherently individualistic, there are apparent counterexamples. Consider the following sentences:

\begin{quote}
\begin{description}
\item[(9)] It's impossible to survive without water for three days.
\item[(10)] It's necessary that a bachelor is unmarried.
\item[(11)] I could have been sick had I eaten the dog food.
\item[(12)] Had I studied harder, I would have known the answer to the exam questions.
\end{description}
\end{quote}

These are modal judgments in my narrow sense. But these are also judgments that seem to be capable of being justified by testimony.

Someone told me that it's impossible to survive without water for three days. So I'm justified to judge that it's impossible to survive without water for three days. Suppose I'm not fluent in English. It appears to make total sense that I'm justified in forming the judgment (10) simply because someone told me about it. Something similar applies to (11) and (12). So apparently, we have counterexamples to the general claim that modal judgments cannot be justified by testimony.

Here's a positive argument that shows that these \emph{apparent} counterexamples are not genuine counterexamples. Although the justification for these judgments \emph{appears} testimonial, there's a case to be made that the justification is in fact not testimonial. Consider the following general claim [T]:

\begin{quote}
\begin{description}
\item[\textbf{[T]:}] For any modal judgment p, when it appears that one is justified to make the judgment that p based on testimony, there's a non-modal judgment q such that (i) the rejection of q doesn't indicate that a person is unqualified to testify on the subject matter of p, (ii) the negation of q is consistent with p, and (iii) had the testifier explicitly \emph{\textbf{rejected}} q while saying p, one wouldn't have been justified to make the judgment p.
\end{description}
\end{quote}

I believe [T] is true. Take (9) as an example. Someone tells me that (9). And I'm justified in forming the judgement that (9). This \emph{seems} to be a textbook case of testimonial justification. But suppose the testifier doesn't simply say (9); suppose she also explicitly \emph{denies} that having no water for three days \emph{causes} death. Then I'm very strongly inclined to say that I'm no longer justified to assert (9) even though the other person keeps telling me that (9) is the case. In this case, (9) is the modal judgment p and the causal judgment being rejected is q.

Notice that the rejection of the causal claim is consistent with the modal judgment (9). And rejecting the causal claim isn't something unreasonable that would disqualify a person as a testifier. Surely testimonial justification is meant to be defeasible. But what undermines a testimonial justification cannot be inexplicably arbitrary. \emph{\textbf{If}} my original justification for (9) was really based on testimony, it shouldn't make any difference that the testifier now rejects the causal claim. I should still be justified to make the modal claim base on testimonial justification given that (i) and (ii) of [T] applies. So, an explanation is called for. That the rejection of the causal claim has this epistemic effect strongly suggests that the real story behind my original apparently testimonial justification for (9) is like the following:

\begin{quote}
When the testifier tells me about (9), I take it that she's implicitly telling me that having no water for three days causes death. And I'm justified to accept the non-modal causal judgment based on what I take to be her implicit testimony. From this non-modal claim, I infer that it's impossible to live without water for three days, i.e., (9). The source of my justification for (9) is the causal claim I get from the testifier, not the testifier's claim that (9).
\end{quote}

This is a good account of what has happened because it explains why I'm justified in accepting (9) when someone tells me about it but the justification for (9) vanishes when the testifier explicitly denies the causal claim --- the causal claim is the true source of the modal justification, \emph{not testimony}.\footnotemark

\footnotetext{It is worth pointing out that even for a reductionists who think that causal claims can be reduced in modal terms, this is a peculiar epistemic phenomenon that has to be explained. Why is it that we cannot be justified to accept (9) via testimony directly? Why does the testimonial justification need to go through the causal claim? Note that even if the causal claim is modal \emph{in some way}, the rejection of the causal claim is still compatible with (9) --- causation is not necessary for the impossibility. And the rejection of the causal claim is still not crazy. This is in fact another epistemological advantage to not view causal claims as reductively modal. By not doing so, we now have an option to explain why testimonial justification has to go through the causal claim and cannot be directly applied to (9): the causal claim is non-modal, (9) is modal, and modal justification cannot be testimonial.}

[T] is true not only for (9), but (10) - (12) as well. The apparently testimonial justification for (10) vanishes if the testifier explicitly denies \emph{that ``bachelor" means ``unmarried man"} (a non-modal claim about word-meaning). The apparent testimonial justification for (11) disappears if the testifier rejects \emph{that eating dog food causes sickness} (a non-modal claim about causation). Similarly, if the testifier denies \emph{that the exam questions are about the study materials} (a non-modal claim about exam content), there is no apparent testimonial justification for (12). Claims about word meanings, causations, and exam content \emph{may entail} modal claims, but they aren't themselves modal in the strict narrow sense.

Just like (9), [T] gives us a compelling reason for thinking that, in all those cases where the justification for a modal judgment appears to be testimonial, the modal judgments are in fact not justified by testimony at all. It only appears that there is testimonial justification going on because one \emph{ends up} making the same judgment as the person who told you about it. Instead, we have implicit testimonial justification for certain non-modal judgments, based on which we infer the relevant modal judgments \emph{on our own}.

I don't have a deductive argument to prove that [T] is true. And the relevant non-modal judgment q has to be determined on a case-by-case basis. But, by showing that and how [T] is plausible in textbook cases like (9) - (12), I hope to have shown the plausibility for thinking that [T] is generally true. Since most general principles in philosophy are motivated by paradigm examples in this way and I'm not aware of any obvious counterexample to [T], I don't think it's a defect that I haven't offered an ``in principle" argument for [T]. If [T] is true generally, all apparent testimonial justifications for modal judgments would be explained away as \emph{justified by inference} based on non-modal judgments like (9) - (12). We end up not having genuine counterexamples to the claim that modal justification has to be individualistic.

\begin{center}
***
\end{center}

There are cases that lend support to the claim that modal judgments cannot be justified by testimony (section 2.2) and there are cases that seem to suggest otherwise (section 2.3). But I've offered a positive argument based on [T] for thinking that the latter cases that appear to be counterexamples to my thesis are not genuine counterexamples. Putting all this on balance, I think we then have a substantive case for the claim that there is no testimonial justification for modal judgments.

We often justify our judgments about actuality based solely on the fact that others say so. But when it comes to judgments about \emph{how things are necessarily}, about \emph{how things could be}, and about \emph{how things could have been had things been different}, we can only rely on ourselves for justification.

No wonder there is an individualistic trend in modal epistemology. That is neither an accident nor a mistake. Whereas it makes sense for a layman to \emph{just trust} someone to form justified beliefs about history or astrophysics or mathematics or logic, it makes little sense to do the same about modality. If so, we need to take this new piece of datum seriously when we theorize about modality.

\section*{3 Explaining away the Datum?}
There are things that people might say against taking the new datum about modal justification seriously. The following utterance sounds misguided:

\begin{quote}
\begin{description}
\item[(13)] I'm angry; I believe that because someone told me so.
\end{description}
\end{quote}

One might try to explain the awkwardness by saying that whether I'm angry or not is the kind of thing that I'm in a very good position to learn about on my own if I try to introspect, which isn't difficult. That is why it's odd for one to form a judgment about my own anger based on someone else's testimony. Explained this way, the awkwardness of (13) does \emph{not} show that our judgments about our own emotion are the kind of judgments that cannot be justified by testimony. They can. It's simply that our unique position to make good judgments about our own emotions generally renders others' inputs \emph{pragmatically} awkward to even be brought up.

One might wonder whether the same is true of our modal judgments. That is, it isn't the case that testimony cannot be justification for modal beliefs; it's only that we all have great equipment at our disposal for modal justification that renders others' input pragmatically negligible. If so, there is no new datum about modal justification to be taken seriously.

To begin with, I don't think that is the right diagnosis of (13)'s awkwardness. Suppose an apple is placed before Susan and me. We are both facing the apple. But whereas Susan is staring at the apple, my eyes are shut. I could very easily open my eyes and see for myself, but Susan's testimony is enough for me to be justified in believing that there is an apple in front of us. So the fact that we have very good equipment at our disposal to access certain truth at ease on our own does not render reliance on others' testimony pragmatically awkward. And the explanation for the awkwardness of (13) has to be refined.

The proper explanation for (13)'s awkwardness isn't that we have good access to our own emotions. Instead, the right explanation is that our access to our emotions is \emph{\textbf{hugely superior in comparison to}} others' access to our emotions. It wouldn't be awkward for me to rely on others if they are at least as good as I am at accessing the relevant truth. But if I'm much better at doing that with ease than everyone else, it would then be pragmatically awkward for me to cite others' testimony as justification. Again, according to this line of reasoning, it isn't that the judgments about my emotions cannot be justified via testimony, but that citing such justification is pragmatically odd given that I have a hugely superior source available to cite. And if this explanation for (13)'s awkwardness also works for explaining the awkwardness of attempting to justify (5) by testimony, there is no new datum for modal epistemology.

But the refined diagnosis of (13)'s awkwardness doesn't apply to our modal judgments. We may have a good epistemic access to modal truths (conceiving, imagining, supposing, intuition, etc., depends on one's modal epistemology). But so does everyone else. The fact that we are in a great position to access certain truth doesn't make others' input worthless unless we are in a \emph{far better} position to access that truth than others'. That's the case for our emotions. But that isn't the case for judgments about modality. Each of us is in a uniquely privileged position to make good judgments about our own \emph{emotions}. Nothing like that can be said about \emph{modality}.

So, whereas the awkwardness of appealing to testimony with regard to our own emotions does not indicate that testimonial justification is not available for judgments about our own emotions because said awkwardness can be pragmatically explained instead, the awkwardness of appealing to testimony for modal judgments doesn't have a similar alternative diagnosis. Hence, it seems that there is a new datum to be taken seriously.

\section*{4 Modal Descriptivism}

Let us consider the following extremely popular view about modal judgments:

\begin{quote}
\begin{description}
\item[\textbf{[Modal Descriptivism]}] Modal judgments are primarily descriptions of facts.
\end{description}
\end{quote}

According to modal descriptivism, when we say that it's possible for Messi to be two inches taller than he actually is, we are making a statement that primarily purports to describe a fact or a state of affair. The function of modal claims like that is like any ordinary non-modal statements, e.g., ``there is a bagel on the table".

Modal descriptivism is a very weak claim. It's even compatible with the view that all claims about possibility and necessity are false because the facts of modality fail to obtain. This is because modal descriptivism only says that modal judgments \emph{purport} to describe facts, it doesn't say that those descriptions are true.

Modal descriptivism is indubitably the dominant view in the field. There is the exotic Lewisian modal realism (Lewis 1986); there is the alternative ersatzist possible world approach (Plantinga 1992; Wang 2015); there are the newcomers --- e.g., the power-based theory (Jacobs 2010) and essence-based theory (Lowe 2012) --- that ditch possible worlds; and there is the deflationary yet descriptivist neo-conventionalism (Cameron 2009; Sider 2011). This is just to name a few major modal metaphysics theories that fall under the descriptivist umbrella. These theories offer different views about what the facts of modality are, but they are facts nonetheless; and modal judgments purport to describe those facts.\footnote{Facts of modality do not have to be modal facts. For a reductionist about modality like Lewis, the facts of modality are non-modal.}

\section*{5 Unpopular Anti-Descriptivism}

Anti-descriptivism about modality (i.e., the negation of modal descriptivism) had its moment during the heyday of logical empiricism. The view has, however, fallen out of favor very quickly. It has few advocates in the contemporary literature. As far as I can tell, three main reasons have been offered to motivate anti-descriptivism: (i) problem about the source of necessity; (ii) pragmatism; (iii) epistemic shaming.

Blackburn (1993: 53-54) challenges modal descriptivism by arguing that it faces a dilemma in accounting for the source or ground of necessity. The challenge, roughly put, is that \emph{either} one has to explain a necessary truth with another necessary truth which leads to an infinite regress that leaves descriptivists with no explanation for why there are necessary facts at all \emph{or} one has to explain a necessary truth with a contingent truth which leads to contradiction.

In response to that, I think Cameron (2010) was right to point out that the dilemma relies on the following four assumptions: (a) explanation of a necessary truth has to be a proposition (137-138); (b) necessary truths are necessarily necessary, i.e. S4 (140); (c) if what explains p weren't the case, p wouldn't be the case (142); (d) when we explain the necessity of p by a necessary truth q, the necessity of q does some of the explanatory work (144). And every single one of these assumptions can reasonably be resisted by the descriptivists, making Blackburn's argument for anti-descriptivism dialectically weak. (Which one to resist depends on the version of descriptivism we are considering; e.g., a reductionist like Lewis surely rejects (d).)

Instead of thinking that there is anything problematic about modal descriptivism in particular, Brandom (2008) is motivated to endorse anti-descriptivism because of his pragmatism as a general meta-philosophical project. By the pragmatists' light, analyzing the meaning of our judgments by laying out the kinds of \emph{fact} those judgments purport to pick out --- presuming that that is what our judgments do --- essentially involves conceiving of a \emph{theoretically} privileged metaphysical language in which the world is carved up correctly and completely into facts. And pragmatism denies that it makes sense of speak of such an absolutely privileged language. For the pragmatists, privileging certain part of our language occasionally is only a strategic move for the sake of illuminating language use:

\begin{quote}
Semantics on this view [i.e., pragmatism] is an inherently Procrustean enterprise, which can proceed only by theoretically privileging some aspects of the use of a vocabulary that are not at all practically privileged, and spawning philosophical puzzlement about the intelligibility of the rest. On this conception, the classical project of analysis is a disease that rests on a fundamental, if perennial, misunderstanding --- one that can be removed or ameliorated only by heeding the advice to replace concern with meaning by concern with use. (Brandom 2008: 7)
\end{quote}

Simply put, the pragmatists' problem with modal descriptivism is that they don't consider the fact-mapping business legitimate at all, modal or not.

I don't wish to take side on this broad meta-philosophical debate. It's true that if Brandom's pragmatism is right, modal descriptivism isn't. Dialectically, however, a super-big-picture and meta-level concern like this carries very little force in a local debate about modal judgment. Analogically: surely, if other mind skepticism is right, testimony cannot be a fundamental source of epistemic justification; but to challenge fundamental testimonial justification by appealing to the skeptical worry about other mind is to address a local debate with a \emph{disproportionately meta-level} concern, which makes it a dialectically weak challenge in a local context. Although pragmatism can be a reason to reject descriptivism about modal judgments, it shouldn't be considered a very compelling reason.

Finally, I think the most salient motivation for anti-descriptivism has been epistemic. If modal judgments are descriptions, we need to be able to give an account of the kinds of fact those judgments describe; and we better be capable of accessing those facts \emph{if} they exist. And some philosophers are pessimistic that a satisfactory account can be offered. For example, the logical empiricists think that the only access we have to the world is empirical, which wouldn't give us access to those facts of modality, even if there were such things. If modal judgments aren't descriptions of facts in the first place, the immediate worry about making sense of our epistemic access to the relevant facts vanishes. It's safe to say that logical empiricism is almost universally abandoned. But a similar epistemic concern about modal descriptivism survives among philosophers who are sympathetic to anti-descriptivism (e.g., Thomasson 2007: 136).

This kind of epistemic shaming against descriptivism is, again, dialectically weak for two main reasons. First of all, such epistemic shaming works, if it works at all, against only certain exotic theories about the facts of modality, e.g., the Lewisian possible worlds. If modal judgments are about facts in some spatiotemporally isolated regions, then it certainly seems legitimate to wonder how it is that we have epistemic access to those facts, which are spatiotemporally removed from us. The same, however, cannot be said of the less exotic metaphysics of modality. For example, if possible worlds are maximally consistent linguistic entities like propositions, there is no obvious epistemological problem. Peacocke (2002), for example, has demonstrated how such an ersatz theory of possible world may handle our modal knowledge. It's dialectically questionable to use an argument to challenge a view when the argument only works effectively against the most controversial version of the view. Hence, it's questionable to use epistemic shaming to challenge modal descriptivism if the shaming only works against the exotic versions of it.

Secondly, the epistemic shaming isn't dialectically as strong as its proponents seem to think it is \emph{even if} we assume that the Lewisian route is the only way to be a modal descriptivist. Assume a certain form of internalism about justification, which says that epistemic justification is about following the rational epistemic norms and such norms are neither necessarily reducible to nor necessarily picked out by any externalist truth-tracking relations (e.g., Pollock 1999; Cruz \& Pollock 2004). If so, a metaphysics that provides exotic truthmakers for our judgments doesn't obviously give rise to epistemological troubles. If truth-tracking is \emph{not necessary} for justification,\footnotemark there is no reason for thinking that exotic truthmakers make epistemology more difficult. Under such a view, epistemology and metaphysics are relatively autonomous and it's not a serious objection against a metaphysics that it cannot offer a metaphysical underpinning of the relevant epistemology. Given that internalism isn't a marginal view at all, the epistemic shaming isn't as fatal as it sounds even against the most exotic modal metaphysics.

\footnotetext{Truth-tracking can still be \emph{sufficient} for epistemic justification.}

So, although Thomasson might be right that descriptivists fail to offer any truly convincing argument against anti-descriptivism, reasons that have been offered to motivate anti-descriptivism haven't been very powerful either. Given the general realist trend in contemporary metaphysics, it's perhaps not surprising that anti-descriptivism about modal metaphysics hasn't enjoyed any genuine revival.

\section*{6 Best Explanation of New Datum}

McGrath (2011) begins with the datum that there is no testimonial justification for moral judgments and argues that that is a strong reason to reject objective moral facts (i.e., moral realism). Now that I have argued that there isn't testimonial justification for modal judgments either, should that give us a reason to reject objective facts of modality (hence rejecting modal descriptivism)? My answer is positive. If we take the new epistemological datum seriously, it provides good evidence for anti-descriptivism about modality because anti-descriptivism provides the best explanation to the new datum. And this evidence for anti-descriptivism is, though certainly not conclusive, dialectically more powerful than the ones that we have examined in section 5.

But before we take a closer look at the sense in which anti-descriptivism offers the best explanation, let's first consider how it can offer any explanation for the new datum at all. I'll focus on one particular form of anti-descriptivism --- modal expressivism:

\begin{quote}
\begin{description}
\item[\textbf{[Modal Expressivism]}] Judgments about non-actual possibility and necessity are linguistic devices to express our evaluative attitudes, not devices to describe facts.
\end{description}
\end{quote}

Different versions of modal expressivism would have different things to say about the kinds of evaluative attitude our modal judgments express. But for the sake of illustration, I'll focus on one particular version of expressivism.

\begin{quote}
\begin{description}
\item[\textbf{[Modal Expressivism+]}] A judgment about the non-actual possibility or necessity of p is a linguistic device to express our \emph{sentimental detachment or attachment} to p, not a device to describe facts.
\end{description}
\end{quote}

So, for example, when I form the judgment that it's \emph{necessary} that bachelors are unmarried, I'm expressing my \emph{sentimental commitment} not to give up ``bachelors are unmarried" in the light of any new information I'll come across. When I judge that it's \emph{possible} for the physical geometry to be Euclidean, I'm expressing my \emph{sentimental openness} to accepting ``the physical geometry is Euclidean" in the light of new information.\footnotemark

\footnotetext{I have no intention to develop the view fully here. One of the potential obstacles standing in the way is that we want to give an expressivist analysis for modal judgments \emph{without} making it all right to judge that \emph{necessarily, I exist} (this wouldn't be a problem for Williamsonians, who believe that everything that exist exists necessarily; but then, if one is on the expressivist track, Williamsonian reasons for endorsing necessatarianism are probably not available or at least not as compelling). Just to gesture at a solution: A way to go is to say that sentences with indexicals (e.g., ``I") express different judgments in different contexts; in other words, ``I exist" is a \emph{judgment-schema}, not a judgment. And modal judgments are expressions of doxastic sentiments towards the judgments expressed in particular contexts, not expressions of sentiments towards judgment schemas. Bruce can reasonably express openness to the idea that Bruce does not exist; for all he know, he could be misled into thinking that he is not Bruce after having an amnesia. So there is nothing wrong in expressing sentimental openness towards the non-indexical proposition expressed in a context by the statement ``I exist". Certainly, so much more need to be said to make this an adequate answer to the problem, but the main goal of presenting Modal Expressivism+ here is not to offer a comprehensive development and defense of the view. That will have to be left for another occasion. The primary goal here is rather to show that an expressivist approach to modal judgements has important resources to account for the new datum that descriptivism doesn't.}

The above analysis works rather smoothly for de dicto modal judgments. But \emph{de re} modal judgments like ``it is possible for this apple not to have the color that it has" is a bit trickier. The following is \textbf{not} what we want to say: that modal judgment expresses my sentimental openness to accepting ``this apple doesn't have the color that it has" --- which is contradictory, making said sentimental openness an irrational sentiment. But there is nothing irrational, presumably, in judging that this apple is red but that it's possible for this apple not to be red. So, in cases of de re modal judgments like this, perhaps the following is what we should say: that modal judgment expresses my sentimental openness to the thought \emph{of this apple}, which I currently take to have a certain color, that \emph{it} in fact has a different color. That way, the expressionist approach can handle de re modal judgments adequately too; they just become expressions of our \emph{doxastic sentiments de re}.

Finally, by the light of this particular version of expressivism, counterfactual judgments of the form ``if it were p, it would be q" aren't judgments that describe facts about the modal connection between p and q. Instead, it's also an expression of doxastic sentiment. When one utters ``if it were p, it would be q", one is simply expressing one's \emph{eagerness to accept} q while \emph{pretending} that p is the case (and ignoring anything that contradicts p directly). Suppose I say ``if Messi hadn't played, Barcelona would have lost". I'm not describing any facts. Instead, I'm pretending that Messi didn't play and ignoring all the information that directly indicates that Messi played (that includes the goals Messi scored) and, \emph{under such pretense}, expressing my eagerness to accept that Barcelona lost given all the information I have. To express sentiments under a pretense is quite common. After all, that is what people do when they watch a movie: they know what is shown on the screen is not real, but they express their genuine sentiments under the pretense that it's real.

What I have offered is indeed a very rough sketch of an analysis of modal judgments under Modal Expressivism+. It isn't my intention here to fully develop and defend the view. For my purpose, a rough sketch that shows how such a view would work in principle suffices.\footnotemark

\footnotetext{Notice that although Modal Expressivism+ introduces epistemic elements into the analysis of modal judgments, it does not thereby collapse metaphysical and epistemic possibilities. It is epistemically possible for me that p just in case p contradicts the things I know. That says nothing about my doxastic sentiments. It can be epistemically possible for me that p even if I am \emph{unwilling} to accept p no matter what new information shows up.}

If modal judgments are expressions of doxastic sentiments as Modal Expressivism+ says, then being justified in making a modal judgment isn't to be justified in making a descriptive claim about reality but to be \emph{justified in having and expressing certain doxastic sentiments}. Since the mere fact that someone else has and expresses certain sentiments doesn't provide any justification for my having and hence expressing the same sentiments at all, there is no such thing as testimonial justification for modal judgments if Modal Expressivism+ is true. Therefore, a view like Modal Expressivism+ can help us make sense of the new datum.

So, some version of anti-descriptivism, e.g., Modal Expressivism+, can give us a decent explanation for the new datum. How about modal descriptivism? As long as modal judgments are descriptions of facts, and as long as we all have epistemic access to those facts at all, there is no good way to make sense of the illegitimacy of testimonial justification; and that's especially problematic dialectically given that testimonial justification is an important kind of justification for all other judgments which are \emph{uncontroversially} descriptions of facts.\footnotemark

\footnotetext{This sets aesthetic and ethical judgments aside. It is at least controversial whether they are \emph{descriptions} of aesthetic and ethical facts. And it is perhaps not an accident that both kinds of judgments \emph{seem} to have a similar anti-testimonial nature (see Robson 2012; Hills 2013). But note that, \emph{even if} aesthetic and ethical judgments \emph{were} anti-testimonial, there could still be substantive differences among these judgments such that we don't have to be descriptivists for either all them or none of them. For example, understanding what morality is might essentially involve understanding that morality is autonomous, which would rule moral deference. The concept of modality doesn't appear to have this feature.}

Since Modal Expressivism+ shows that anti-descriptivism has the option of accommodating the new datum and modal descriptivism does not, the new datum speaks in favor of anti-descriptivism via an inference to the best explanation. The power of this evidence for anti-descriptivism can be better appreciated once we notice that it speaks against facts of modality in general \emph{regardless of the metaphysical status of those facts}, unlike the epistemic shaming we talked about earlier. That is, those facts could be mind-independent, mind-dependent, about relations among abstract concepts, even about a fiction about possible worlds; it doesn't matter what those facts are supposed to be, as long as modal judgments are descriptions of facts, testimonial justification \emph{should} be legitimate. E.g., I can be justified in holding beliefs about certain \emph{fictional facts} in a movie that I'm yet to see based on someone's spoilers alone.

Of course, anti-descriptivism, and especially expressivism, about modality has its own problems. For example, if modal judgments are \emph{just} expressions of doxastic sentiments, those judgments are arguably not truth-apt (unless one has a more relaxed conception of truth, see Wright 1995 for example). If modal judgments aren't truth-apt, among other things, we face the Frege-Geach problem; namely, we have difficulty making sense of the compositionality of and the inferences we draw with modal judgments (e.g., Hale 1993; Schroeder 2008b). For that reason, I'm \emph{not} saying that the new datum, if taken seriously, \emph{proves} anti-descriptivism \emph{conclusively}. Nonetheless, the new datum provides defeasible but compelling support for anti-descriptivism because, whereas we at least have more or less a sense of where to look for solutions to issues like the Frege-Geach problem given that similar problems have been the center of discussion elsewhere, e.g., in meta-ethics (e.g., Blackburn 1984; Schroeder 2008a), it isn't at all clear where the descriptivists can even begin to accommodate the new datum.\footnotemark The new datum makes a much more compelling case for the view than what we are given in the literature. What I hope to have achieved is to put the view back on the negotiation table with all the chips it actually has so it can at least have a fair run in the debate.

\footnotetext{Sure they can find a way to explain \emph{away} the new datum. But I am saying there is no way out for descriptivists \emph{if} my earlier argument succeeds and we are supposed to take the new datum seriously.}

\section*{7 Modal Normativism}
I want to wrap up my discussion with a cautionary note. I said that the new datum supports anti-descriptivism because at least some version of it can explain the new datum but it doesn't seem likely that any descriptivist view can accommodate the datum. Not all versions of anti-descriptivism can do that though. And I want to show how a version of anti-descriptivism recently defended by Thomasson --- modal normativism --- \emph{arguably} fails to handle the new datum.

Thomasson argues that modal judgments are devices we used to \emph{demonstrate} the linguistic norms that govern us. Consider the modal judgment \emph{that it is necessary that water is H\textsubscript{2}O}. That is a conclusion we draw from two other judgments: (i) H\textsubscript{2}O is the substance that plays the water-role actually; (ii) water is necessarily whatever substance that plays the water-role actually. The former is a non-modal, natural scientific judgment. I'm not interested in that. Judgment (ii) is, however, a modal judgment. By making that judgment, according to Thomasson, we perform not an act of description but an act of demonstration --- demonstrating a \textbf{semantic rule} that governs our language, the rule that says the term ``water" applies to whatever chemical substance that plays the water-role. The modal judgment doesn't describe the rule; instead, it \emph{demonstrates} the rule. We can demonstrate the rules for making a good omelet by flipping a piece of plastic in a toy pan, and we aren't making any description by doing so. Similarly, by making modal judgments like (ii), we don't purport to describe any facts --- not even the normative facts (2007: 140-141).

Modal normativism is a version of anti-descriptivism because modal judgments aren't primarily descriptions according to the view. But it isn't expressivism either --- although Thomasson had once called the view expressivism, I believe she rightly retracted the label. That is because the demonstration of norms isn't \emph{merely} an expression of evaluative attitude under her view. Modal normativism is compatible with modal judgments' being truth-apt even with a robust notion of truth in play (ibid: 148-149).\footnotemark

\footnotetext{I do not recall reading her explicit explanation of retracting the label ``expressivism". So this is my rationalization on her behalf. She explicitly says that modal judgments are truth-apt in a robust way.}

Modal normativism has many good features. However, our new datum stands in its way. It isn't clear that modal normativism has the resources to fully accommodate the illegitimacy of modal justification based on testimony.

Suppose I arrive in a foreign country. It's a matter of social norm that people in this country queue in a zig-zag line instead of in a straight line. Their actions --- queuing in a zig-zag line --- demonstrate the social norm that governs the people there. Simply by seeing the way people queue, I'm justified in following suit and showing my acknowledgment of the same social norm in the same country. This kind of scenario happens to travelers a lot.

Moving from norms for queuing to norms for judgment, I think the same is true by parity of reasoning. Suppose Thomasson's view is true that, when people say ``water is necessarily whatever substance that plays the water-role actually", they aren't primarily performing an act of description but an act that demonstrates a norm of their language. On hearing \emph{many} people make that modal judgment, and if that's a norm demonstration, I \emph{should} be prima facie justified to make the same judgment to demonstrate the same linguistic norm since those people and I share a language. So, if modal normativism were true, I would be justified in judging that water is necessarily whatever substance that actually plays the water-role \emph{just because others directly say so}. And that runs against the new datum (see also my discussion about applying [T] to (10) in section 2.3).

\section*{8 Conclusion}

I begin the inquiry by bringing attention to the individualistic trend in contemporary modal epistemology, which is at odds with contemporary epistemology generally. By eliciting our intuitions about various kinds of modal judgments, I try to motivate the view that the individualistic trend is an indication of a positive feature about modal justification instead of a flaw of contemporary modal epistemology. I then argue that if we agree to take the positive feature seriously, the feature provides us with new evidence for anti-descriptivism about modality. For those who are firmly committed to modal descriptivism nonetheless, the systematically individualistic nature of modal justification remains a mystery to be explained.

\nocite{*}
\bibliographystyle{theapa}
\bibliography{Modal_Testimony}


\end{document}