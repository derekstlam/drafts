\documentclass[a4paper,12pt]{article}
\usepackage{geometry}
\geometry{top=1.3in}


%opening
\title{\textbf{A Pessimistic Induction about the Objective Presentness}}
\author{Derek Lam}

\begin{document}

\maketitle

\begin{abstract}
Some philosophers argue that non-presentist A-theories problematically implies that we do not know that this moment is present. The problem is generally presented as arising from the unholy combination of the A-theoretic ideology of a privileged presentness and a non-presentist ontology. The goal of this essay is to show that the epistemic problem can be rephrased in terms of a pessimistic induction. By doing so, I will show that the epistemic problem, in fact, stems from the A-theoretic ideology alone. Hence, once it is properly presented, the epistemic problem presents a serious threat to all A-theories.
\end{abstract}

\section*{1 The Epistemological Question}

Here's an observation: There were many people before us; they all believed that their moment, instead of ours, is the present (they didn't believe that their moment \emph{was} present). This observation has led philosophers to believe that there is an epistemic tension in non-presentist A-theories, namely, those theories that say that the tenses are real and \emph{at least} the past still exists.\footnotemark Suppose ``present" isn't a pure indexical like ``here" but denotes an objective feature of a special moment. The overwhelming majority in the history of mankind had beliefs about the location of presentness that are false (even though they \emph{were} true at that time). An epistemological question inevitably arises: if all those beliefs about the location of presentness before us are false, does that not give us a reason to think the same about our beliefs on the same topic? For some philosophers, this questions the plausibility of the attempts to combine an A-theoretic commitment to an objective presentness and a non-presentist temporal ontology (namely, the non-presentist A-theories).

\footnotetext{Theoretically, to be a non-presentist A-theorist, one just have to be an A-theorist who accepts the reality of the tenses and also think that some moments other than the present one exists. So, a non-presentist A-theorist can defend the view that only the future and the present moments exist and \emph{not the past}. I define non-presentist A-theory in the way I do here because the growing block view and the moving spotlight view are the only two non-presentist A-theories being defended and taken seriously in the literature. I want to focus on them and set aside other theoretical possibilities.}

The primary goal of this essay is to argue that, contrary to traditional wisdom, the aforementioned epistemic tension resides solely in the A-theoretic ideological commitment to an objectively privileged present. \footnotemark The tension has nothing to do with our temporal ontology. Hence it's a mistake to think that there is an epistemic problem only for the non-presentist A-theories. There is a serious epistemic problem for all A-theories.

\footnotetext{Here I am using Quine's (1951) famous distinction between a theory's \emph{ontological} and \emph{ideological} commitment. The ontological commitments of a theory are the objects it posits; the ideological commitments are the basic unanalyzable notions the theory introduces. The A-theories are committed to a \emph{primitive} notion ``present" that represents an objective presentness.}

Here's how I will proceed. In section 2, I'll introduce the epistemic problem for the non-presentist A-theories as it is standardly presented. In section 3, I'll argue that Cameron's (2015) recent attempt to generalize the problem for all A-theories fails. In section 4 - 6, I'll offer a new argument for the conclusion that the epistemic problem challenges all A-theories. To do that, I'll introduce a novel way to flesh out the epistemic problem that renders the ontology of time irrelevant. Finally, in section 7 - 8, I'll address two apparent solutions to the epistemic problem as I present it.

\section*{2 The Skeptical Problem}
It has long been pointed out that there is an epistemic tension in the non-presentist A-theories. The tension has been developed in different ways, but it has almost always been understood as a skeptical problem. That is, the problem is taken to be that, based on the fact about past people's beliefs, if one accepts non-presentist A-theory, then one has to accept that ``nobody ever knows which time is absolutely present." (Russell 2015: 3) --- in particular, we do not know that this moment is present.

The problem has been fleshed out in two ways. Very often, the skeptical problem is articulated in terms of the evidential indistinguishability of people before us and ourselves (Braddon-Mitchell 2004, Heathwood 2005, Merricks 2006). If the non-presentist A-theory is true, there are people before us who believe that, say, 1985 is the present year. Those beliefs are false. But those people are in a similar epistemic situation as I am from the first person's perspective as far as the location of the privileged presentness is concerned. So, from what I have evidential access to, I cannot tell the difference between those people before me and myself. As a result, I am not justified in believing that I am right and those people are wrong when I believe that this moment is present --- that is, I have no epistemic resources to rule out the possibility that I am one of those people who are stuck in the past and that this moment is in fact not present. Justification is necessary for knowledge. Thus, I do not know that this moment is present if non-presentist A-theory is true.

Others try to develop the epistemic tension in terms of the \emph{safety requirement} for knowledge instead (Cameron 2015; Russell 2015). The idea is that if a person knows that p, that person would still know that p at all close enough possible worlds. Given that all the past people still exist and are ``perfectly preserved" in the past, a world at which this moment is a tiny bit earlier than the objectively present moment with everything else remaining the same is a close enough possible world at which I believe but do not know that this moment is present (because that belief wouldn't be true). Therefore, if both the non-presentist A-theory and the safety principle for knowledge are true, we don't know that this moment is present.

Here is a fact: we certainly know that this moment is present. Therefore, either way, if the non-present A-theories entail that we do not know that, there is a problem with the views. This is the standard understanding of the epistemic problem for non-presentist A-theories.

Contrary to the standard understanding, it is worth mentioning that there are philosophers who think that the epistemic tension is in fact not about our knowledge of the location of the privileged present. Miller (2016: 320-321), for example, argues that the \emph{real} problem is that the non-presentist A-theorists cannot account for how we get to have the \emph{justified} belief that this moment is the privileged present given their metaphysic.

I disagree. Surely we can formulate the epistemic problem in the following way: if some form of non-presentist A-theory is right, we have no justification for believing that this moment is present. That would be a problem for the non-presentist A-theories. But \emph{why} is it a problem? The answer is that we clearly have knowledge that this moment is present \emph{and} knowledge requires justification. So, I think it is a mistake to think that the real problem is not about knowledge because it is about justification. I suppose Miller wouldn't be satisfied if the non-presentist A-theorists answer her concern by offering an account of how we come to be justified in believing that this moment is present while admitting that that justified belief does not constitute knowledge.

Furthermore, Russell's (2015) formulation of the epistemic problem is about the safety requirement \emph{for knowledge}. Hence, the challenge is directly about the non-presentist A-theorists' alleged inability to accommodate our knowledge of this moment's being present. If one accepts some form of reliabilism about justification, then, by violating the safety requirement for knowledge, one's belief could also lack reliability to count as being justified. But the problem as Russell formulates it does not have to (and hence should not) be held hostage to a substantive theory of justification, namely reliabilism. As long as one accepts that there is a safety requirement for knowledge, there is a problem for the non-presentist A-theories --- whether or not we also accept reliabilism and think that there is also a problem about justification. So, not only is it not obvious to me that the problem is about justification instead of knowledge, I also find it debatable that the problem is primarily about justification.

Due to these considerations, I will stick to the standard understanding that the epistemic tension is a skeptical problem about knowledge. It is a problem that says, if some version of non-presentist A-theory is true, we do not have knowledge that this moment is present; and we clearly know that this moment is present. Hence, we have a \emph{reductio}.

\section*{3 Skeptical Problem for \emph{Presentism}}

A non-presentist A-theory has an ideological component and an ontological component. The ideological component is about accepting a privileged presentness as an objective feature of reality. The ontological component is the claim that not only the present moment but at least the past moments also exist. As we have seen, the traditional presentations of the skeptical problem capitalize on the idea that this particular blend of ideology and ontology implies that people before us exist and have very different beliefs about the location of the privileged presentness, despite experiencing the world in the same way as we do. There is trouble either because those people have the same kind of evidential access as we have, or because we could very easily be one of them. Presented as such, the problem is meant to target the non-presentist A-theorists specifically.

In his recent book, Cameron (2015) challenges this orthodoxy. He argues that the skeptical problem is either not a problem for the non-presentist A-theories, or it is at best a dialectically weak problem for all A-theories, presentism and non-presentism alike. Here is basically his reasoning:

\begin{quote}
\begin{description}
\item[{[1]}] The epistemic problem is to be understood either in terms of evidential indistinguishability or the safety requirement for knowledge.
\item[{[2]}] If the epistemic problem is understood in terms of evidential indistinguishability, it requires a very stringent internalist conception of justification that would make the epistemic problem applicable to presentism as well. (ibid: 24)
\item[{[3]}] If the epistemic problem is presented with a very stringent internalist conception of justification, its challenge would be dialectically weak.
\item[{[4]}] The safety requirement for knowledge does not create an epistemic problem for non-presentist A-theories at all, according to the right semantics for tensed counterfactuals. (ibid: 40)
\item[{[C]}] The epistemic problem is either a dialectically weak problem for all A-theories or it is not a problem (for non-presentist A-theories) at all.
\end{description}
\end{quote}

On the one hand, \emph{if} Cameron is right about the semantics for tensed counterfactuals (according to [4]), the epistemic problem cannot be fleshed out in terms of safety. On the other hand, \emph{if} he is not right, we can formulate an epistemic problem for the non-presentist A-theories that does not apply to presentism. Since Cameron's goal is to show that there is no serious problem for the non-presentist A-theories, he needs [4] to be true. On the contrary, my goal is to argue that the epistemic problem poses a genuine threat to both presentist and non-presentist A-theories --- I am not trying to defend non-presentist A-theories. And my main beef about the argument [1] - [C] is not [4], but [1] and [2]. So whether [4] is true or not is insignificant for my purpose. Even if Cameron turns out to be wrong about [4], that only means the epistemic problem can be presented narrowly to target non-presentist A-theories specifically. That does not contradict my contention that the epistemic problem is in fact a general problem for all A-theories. I have nothing interesting to say about the semantics of tensed counterfactuals. Therefore, I will assume Cameron is right about [4] and set the epistemic problem formulated in terms of the safety principle aside in the rest of this essay.

I think [2] is false. Cameron believes that [2] is true because of the similarity he sees between the epistemic problem and the problem of external world skepticism. In the case of external world skepticism, the issue is that there is no evidence that we have access to that allows us to rule out the \emph{\textbf{epistemic possibility}} that we are BIVs. Cameron thinks that we should understand the epistemic problem against non-presentist A-theories in the same way: if some form of non-presentist A-theory is true, we do not have evidential resources to rule out the \emph{\textbf{epistemic possibility}} that we are stuck in the past just like the people exist before us. Since the problem is about epistemic possibility, the fact that presentism is true or even metaphysically necessary would not help, for that metaphysical thesis alone would not rule out unwanted epistemic possibilities (ibid: 25).

Let me begin by pointing out that, if Cameron was right that the epistemic problem is about our lack of evidential resources to rule out the epistemic possibility that we are stuck in the past, the problem wouldn't threaten only the A-theories. It threatens the B-theories too. Suppose we are to understand the epistemic problem in the mold of external world skepticism: just as the presentists cannot appeal to the alleged truth of presentism because that presentism is false is an epistemic possibility left open by the mere truth of presentism, the B-theorists also cannot appeal to the B-theoretic idea that being present is not an objective feature of reality in order to rule out the epistemic possibility that we are stuck in the objective past given that we are yet to rule out the epistemic possibility that the B-theory is wrong about that. What this shows, I fear, is that understanding the epistemic problem in the mould of external world skepticism fails to capture what the epistemic problem is really getting at and mischaracterizes the challenge it posts.

More importantly, when the epistemic problem is presented in terms of evidential indistinguishability, it says that, assuming some form of non-presentist A-theory, there \emph{are} people who have the same sort of evidential access as we have and believe that this moment of ours is not present; and that should force us to be skeptical of our belief that this moment is present.

Consider the following analogous scenario. Suppose a group of people have dinner together. When the check arrives, we all take a quick look at it. I think each of us has to pay \$22. But it turns out that everyone else thinks that it is \$26 per person. Presumably, given that there are so many \emph{actual} people who come to a different conclusion based on the same evidence, I should not claim that I know that each of us has to pay \$22.

If we were to apply what Cameron says about the epistemic problem to the dinner case, basically he would be saying that we should understand my skeptical reasoning in the situation in the mold of external world skepticism: surely I should not claim that I know that each of us has to pay \$22; but whether there actually are people who think that it is \$26 per person is not the point, what is crucial is merely the fact that it is epistemically possible that there is someone at the table who looked at the check and concluded that it is \$26 per person.

I think this is clearly not the point of the original reasoning in the dinner case. Surely if I were a radical skeptic who endorsed some extremely demanding criterion of knowledge, I would have some radically skeptical reason to deny knowing that the meal is \$22 per person even if no one actually thinks otherwise. But we don't need to be radically skeptical to find the self-doubt in the dinner case intelligible. In the original reasoning, the fact that people who \emph{actually} think that the meal costs \$26 per person exist is crucial to the conclusion that I should not claim I know that it is \$22 per person. I \emph{wouldn't} have drawn the same skeptical conclusion if I hadn't thought that some people \emph{actually} think that the meal costs \$26 per person, even if it was \emph{epistemically possible} that they think so secretly. Thus, it would be a distortion of the original reasoning to insist that what is really going on in the dinner case is only about the mere epistemic possibility that someone might think that the meal costs \$26 per person.

My view on Cameron's [2] is the same. Assuming that some form of non-presentist A-theory is true, there are many people who believe that this moment is not present base on the same kind of evidence as we do. When philosophers argue that that undermines our claim to know that this moment is present, the argument is the same in spirit as the one in the dinner case. Certainly one can reach the same skeptical conclusion on a radically skeptical basis (i.e., that the epistemic possibility of otherwise alone is sufficient for skepticism). But there is no basis for insisting that the epistemic problem has to be understood in the mold of radical external world skepticism --- doing so is quite uncharitable especially when we agree with Cameron that the epistemic problem is dialectically weak presented this way.\footnotemark

\footnotetext{That is, given [3], interpreting the epistemic problem in the mould of external world skepticism is rather uncharitable even if one remains unconvinced that interpreting the problem this way threatens the B-theory as well. I believe the dinner example should be able to show that.}

\emph{If} we choose to understand the skeptical challenge to non-presentist A-theories as a radically skeptical one, then it is indeed a challenge to all A-theories. Cameron is right about that. And the challenge would indeed be a dialectically weak one given that there is no strong reason to accept such a demanding criterion of knowledge that requires ruling out all alternative epistemic possibilities based on the evidence one has access to. But Cameron fails to offer a convincing case for [2], which requires us to read the epistemic problem uncharitably in the mold of the argument for external world skepticism.

\section*{4 A Pessimistic Induction}

I agree with Cameron's conclusion to a certain extent: the epistemic problem is a problem for all A-theories (but not the B-theory). However, I think his argument for the conclusion is flawed, as I have argued. In the rest of this paper, I will offer a new argument for the conclusion (one that involves rejecting [1] as well). But whereas he aims to defend non-presentist A-theories by showing that, if there is an epistemic problem, it is at best a weak problem for all A-theories (due to [2] and [3]), I aim to raise a genuine epistemic challenge to all forms of A-theory.

Here is my strategy. In the rest of this section, I will offer a novel way to flesh out the epistemic problem. It does not appeal to evidential indistinguishability, nor does it rely on the safety principle (hence rejecting [1]). In section 5 and 6, I will show that the way I flesh out the epistemic problem renders the ontology of time irrelevant. Hence, under my way of presenting the epistemic problem, it is a threat to all A-theories, presentist and non-presentist alike. I will then argue that, given that there is a way to flesh out the epistemic problem independent of temporal ontology, trying to solve the epistemic problem by appealing to a presentist ontology is trying to solve the problem in the wrong way: it is to address a problem by targeting something contingent on a particular presentation of a problem instead of dealing with the problem itself.

People before us held beliefs about the location of presentness, e.g., my grandfather once believed that 1967 is the present year. Let's name those beliefs B\textsubscript{1}, B\textsubscript{2}, B\textsubscript{3},\textellipsis. I take it that it's unproblematic to name things before us, regardless of one's ontology of time. After all, we should be able to name Napoleon (I just did!) no matter what we think about the ontology of time. Suppose the non-presentist A-theorists are right that the presentness is an objective property. And there's only one privileged present. So B\textsubscript{1}, B\textsubscript{2}, B\textsubscript{3},\textellipsis \emph{\textbf{are}} almost all false; e.g., my grandfather's belief that 1967 is the present year is not true. Since the beliefs B\textsubscript{1}, B\textsubscript{2}, B\textsubscript{3},\textellipsis are not randomly put together (they are beliefs on the same topic), \emph{by induction}, we have reason to think that \emph{my belief} on the same subject matter, namely, my belief that this moment is present (let's name this belief B\textsubscript{0}), is also false. Notice that this is analogous to the pessimistic induction against scientific realism: when we look at the history of science, most of the theories that were once accepted turn out to be false; so, by induction, it seems that we should think that our current theories are also false. Let me call this inductive reasoning about the truth-value of B\textsubscript{0} the \textbf{pessimistic induction}.

Whereas B\textsubscript{1}, B\textsubscript{2}, B\textsubscript{3},\textellipsis \emph{are} almost all false, they \emph{used to be} true. My grandfather's belief that 1967 is present \emph{is false} but it \emph{was true} when it was 1967. So, there is another induction in the neighborhood. The past beliefs B1, B2, B3,\textellipsis \emph{were} all true. Hence, by a \emph{different} induction, my belief B\textsubscript{0} is true. Past beliefs about the location of presentness were true (even though they are false now), so we have reason to believe that our beliefs about the location of presentness are true. So this moment is present. Let me call this the \textbf{optimistic induction}.

The pessimistic induction is based on how most of those beliefs \{B\textsubscript{1}, B\textsubscript{2}, B\textsubscript{3},\textellipsis\} \emph{are}. And the optimistic induction is about how most of those beliefs \emph{were}. When we want to figure out how things are, an induction based on how things are simpliciter overrides an induction based on how things were. Here is an example to illustrate why this claim is true. I've eaten oranges my entire life. Oranges used to be sweet. But things have changed. Oranges these days are sour. I am wondering whether the slice of orange I am about to put in my mouth is sweet or sour. There are two inductions to be had here. Based on the fact that oranges \emph{have always been} sweet, I have an inductive reason to believe that this orange is sweet. This is an optimistic induction. But knowing that oranges \emph{are} all sour, I also have an inductive reason for believing that this orange is sour. This is a pessimistic induction. When I am interested in finding out whether this orange is sweet or sour, it seems obvious that the pessimistic induction based on how oranges are overrides the optimistic induction based on how oranges were. Similarly, when we are interested in finding out whether my belief B\textsubscript{0} is true, the pessimistic induction overrides the optimistic induction and the fact that almost all of B\textsubscript{1}, B\textsubscript{2}, B\textsubscript{3},\textellipsis used to be true would not help undermine the pessimistic induction which concludes that my belief that this moment is present is false, i.e., that this moment is not present.

The pessimistic induction I have just introduced sounds similar to the reasoning based on a principle Cameron calls \emph{Statistical Knowledge}: ``If you believe p on basis E then, if most of the people who believe p on basis E believe falsely, you do not know p on basis E". (2015: 43) The crucial difference, however, lies in this. For Cameron, the principle Statistical Knowledge is just another way to capture the safety requirement for knowledge: ``I think [Statistical Knowledge] only sounds appealing because [the safety requirement] is appealing". (ibid: 43) On the contrary, the rational pull of my pessimistic induction does not rely on the modal character of knowledge, e.g., the safety requirement. All I need for the pessimistic induction about the objective presentness to work is the innocent idea of the legitimacy of induction about the objective features of things. The rational appeal of induction does not stem from the safety principle; even a modal skeptic can find induction reasonable.

Let me be clear: I do \textbf{not} endorse the pessimistic induction. Obviously, we know that this moment is present. Obviously, to doubt that this moment is present based on the pessimistic induction is absurd. As a result, we have the epistemic problem in the form of a \emph{reductio} argument. If presentness were an objective feature of some special moment, i.e., if the A-theory were true, it would be an objective fact of the matter that B\textsubscript{1}, B\textsubscript{2}, B\textsubscript{3},\textellipsis are all false. Since drawing conclusions about objective features of things by induction is a legitimate way of reasoning, the A-theory would leave us no choice but to accept that the pessimistic induction is legitimate and, as a consequence, that we don't know that this moment is present. This consequence is absurd. By reductio, we should conclude that presentness is not an objective feature of reality, i.e., the A-theory is false.

The force of the reductio can be made even more pronounced by noting that, if the B-theory were true instead, B\textsubscript{1}, B\textsubscript{2}, B\textsubscript{3},\textellipsis are all true because ``present" would be a pure indexical. So, there won't be a legitimate pessimistic induction by the light of the B-theory. It is only when ``present" is taken to denote an objectively privileged moment as the A-theories have us believe, B\textsubscript{1}, B\textsubscript{2}, B\textsubscript{3},\textellipsis are all false.\footnotemark As long as that is the case, we have no choice but to embrace the pessimistic induction. Unlike the B-theory, the A-theory lacks the theoretical resources to explain in a principled manner why the pessimistic induction is no good.

\footnotetext{Or \emph{\textbf{almost}} all false, if we do not wish to assume that this moment is present and that none of the moments before us is present.}

\section*{5 Forrest's Solution}
In response to the standard epistemic problem, Forrest (2004) argues that, although there are past moments and past people, people in the past are no longer conscious and no longer have beliefs. \emph{Technically}, I think this is a legitimate response to the epistemic problem as it has been standardly presented. And the unique force of unpacking the epistemic tension by appealing to a pessimistic induction as I illustrated in the previous section can be better appreciated once we notice that the A-theorists \emph{cannot} block the pessimistic induction by saying that the beliefs B\textsubscript{1}, B\textsubscript{2}, B\textsubscript{3},\textellipsis don't exist anymore. Here is why.

No matter what one's considered metaphysics of time ends up to be, one has to allow induction about things simultaneous with us to be based on things before us. That is how we learn from the past. So, \emph{regardless of our ontology about past beliefs}, we have to allow induction about our presentness locating beliefs at this moment to be based on features of the presentness locating beliefs before this moment (i.e., B\textsubscript{1}, B\textsubscript{2}, B\textsubscript{3},\textellipsis). B\textsubscript{1}, B\textsubscript{2}, B\textsubscript{3},\textellipsis are false. Thus, I conclude inductively that our presentness locating beliefs are also false. The fact that we want induction to reach out to things located at times before us regardless of temporal ontology makes the pessimistic induction immune to responses that are legitimate answers to the epistemic problem as it is traditionally formulated.

Let's give the claim ``B\textsubscript{1}, B\textsubscript{2}, B\textsubscript{3},\textellipsis are false" a closer look. I believe the truth values of beliefs that are about objective features in the world change according to changes of those objective features themselves. For example, I believed that my phone is fully charged. That belief was true. But after a while, my phone is dead and so \emph{that belief} is no longer true. That my belief is no longer true in this case is so plausible that I'm inclined to say, if any theory implies that my belief in fact remains true despite that my phone runs out of battery after a while, that theory has to be mistaken.

So, for instance, suppose someone thinks that (i) propositions have their truth values eternally, (ii) beliefs' contents are propositions, and (iii) beliefs' truth values are directly inherited from the truth-values of the propositions which are their contents. She then has to think that my belief that my phone is fully charged \emph{remains true} even when my phone is dead after a while. This can't be right. This means one should not accept (i) - (iii) all at once. And for the same reason, whatever one's theory of belief content might be, I think it should be beyond question that the beliefs B\textsubscript{1}, B\textsubscript{2}, B\textsubscript{3},\textellipsis \emph{are} false \emph{\textbf{if}} we think that presentness is an objective feature of reality that ``moves".

The pessimistic induction is very similar to the following ordinary induction. I believe that I'll never want to have kids. But most young people's beliefs on the topic are false. Thus, I have an inductive reason for self-doubt, i.e., a reason for not taking my belief seriously. I do not know that I'll never want to have kids. Whereas the self-doubt about having kids in such a case is reasonable, it is not reasonable to doubt that this is the present moment base on the pessimistic induction. Surely we need to classify things right to do induction, as we learn from Goodman's riddle. But the kind of induction that I rely on here is quite ordinary --- it is an induction based on a group of beliefs on the same subject matter. And such an ordinary induction should work whether or not Forrest is right that all those beliefs before us no longer exist.\footnotemark

\footnotetext{That the beliefs are on the same subject matter is crucial for the induction to work. For example, if B\textsubscript{1}, B\textsubscript{2}, B\textsubscript{3},\textellipsis were just random false beliefs about different things, I couldn't infer inductively that my belief on some random topic is also false.}

\section*{6 Threat to Presentism}

I've shown that Forrest's attempt to solve the epistemic problem does not work if the problem is articulated in terms of a pessimistic induction. My comment on Forrest's response to the skeptical problem serves a further important purpose: it is for exactly the same reason that the presentist ontology would not help an A-theory circumvent the epistemic problem. Thus, the problem remains whether one defends a presentist or a non-presentist ontology.

In the dinner case I used earlier, for me to be skeptical that the meal costs \$22 per person, there must be actual people who believe that it costs different, e.g., \$26 per person. That such a person is epistemically possible is not enough. If presentism is right, no past moments, past people, and past beliefs \emph{exist} to require us to retract our claim that we know that this moment is present. Not only that, presentists typically believe that presentism is necessary. Cases of past people that are indistinguishable from ourselves with respect to the kind of evidence we have access to \emph{couldn't possibly exist} to make our belief that this moment is present unjustified or unsafe. The existence of past beliefs is crucial to the skeptical problem as it is traditionally presented.\footnotemark That's why Forrest's non-presentist A-theory, which denies the existence of past beliefs, is an intelligible response.

\footnotetext{Cameron denies this. But as I have argued, I think he is mistaken.}

But as my earlier remark on Forrest's view shows, the pessimistic induction doesn't depend on the existence of past moments, past people, and past beliefs at all. It does not even rely on their existence's being possible. The pessimistic induction stems from certain present actual \textbf{truths} about the past, \emph{not existence} of the past. The following claim is true regardless of our ontology of time: My grandfather's belief about the location of presentness is false.\footnotemark That should be the case even if we think that, necessarily, the present moment is the only moment that exists and, necessarily, the past beliefs no longer exist. As long as we have truths about the past, the pessimistic induction should work even if presentism is true. And that's exactly the point of unpacking the epistemic tension by appealing to induction --- the legitimacy of induction based on past objects should be independent of one's ontology of time.

\footnotetext{``B\textsubscript{1}", ``B\textsubscript{2}", ``B\textsubscript{3}",\textellipsis refer to beliefs before us. The claim ``B1 is false" is true. There has been a standing problem for presentism with respect to apparently true claims that refer to past objects. Presentists deal with the problem in various ways. I don't want to go into the details; but the bottomline is, however the presentists choose to handle the issue, I believe their solution should \emph{not} end up saying that there are no present tense truths about past objects. It has to be true, for example, that Napoleon is no longer present. So I am setting aside Markosian's (2004) controversial suggestion that claims about past objects are in fact all false.}

As I mentioned earlier, this pessimistic induction about the privileged present is in many ways analogous to the pessimistic induction against scientific realism. If we don't think that presentism, which implies that the past scientists no longer exist, is a legitimate response to the pessimistic induction for the scientific realists, it shouldn't be a legitimate response to my version of the epistemic problem for the A-theorists either.

Here let me borrow something from Peter van Inwagen's argument for incompatibilism. Van Inwagen presents three versions of his argument for incompatibilism in his essay on free will (1986). The reason for presenting three versions of the same argument is so that he would be in a position to show that, if an objection against his argument works for only some but not all three ways of presenting his argument, that objection only challenges something contingent on certain way in which his incompatibilist argument is presented, not the argument itself.

Back to our case: by showing that we can unwrap the epistemic tension in terms of a pessimistic induction, I show that the epistemic tension can be understood in a way that has no bearing on the ontology of time. I grant that presentism is technically a legitimate response to the epistemic problem when it is presented in the standard way. But what I have shown is that such a response works only under some but not all presentations of the epistemic problem. Thus, attempts to deal with the epistemic problem by appealing to an ontology of time, even if they succeed technically, are solving the problem in the wrong way. That is, they are not solving the problem by addressing the epistemic tension itself but simply by targeting something peripheral to the problem.

Finally, presentists may be tempted to challenge the pessimistic induction's conclusion by appealing to presentism and their indubitable knowledge that they exist. But appealing to presentism wouldn't help. The epistemic problem says that, if some form of A-theory is true, we do not know that this moment is present. Since we know that this moment is present, no A-theory is true. Notice that the reductio begins with ``if some form of A-theory is true", not ``if some form of A-theory is \emph{\textbf{known}} to be true". From the fact that I \emph{know} that I exist and the \emph{assumption} that presentism is true, I cannot conclude that I \emph{know} that this moment is present. Analogy: from the fact that I know that I exist and the \emph{assumption} that I am a dinosaur, I cannot conclude that I \emph{know} that some dinosaur exists. The knowledge that I exist can only help challenge the pessimistic induction if we also assume that I \emph{know} that presentism is true. But no one in this context should assume that.

What is the point of presenting the epistemic problem? So we know that non-presentist A-theories can't be true. In fact, if I am right about the pessimistic induction, the epistemic problem is presented \emph{\textbf{so we know}} that A-theories simpliciter cannot be true. If that is the goal of the challenge and the point of contention, surely it is dialectically misguided to try to defuse the epistemic problem by relying on the \emph{assumption} that we already know that presentism (and hence some form of A-theory) is true.

Quite the contrary, say the epistemic problem shows that we cannot know that this moment is present if there is a privileged present, and say we know that we exist at this moment. That should lead us to conclude that we cannot know that presentism is true. So, if presentism is true, we cannot know that presentism is true.

The epistemic problem is hence a general threat to all A-theories, contrary to the traditional wisdom that takes the problem to be a challenge only to the non-presentist A-theories, which remain to be minority views. Typically, the epistemic problem of the A-theoretic ideology is presented with a non-presentist ontology. It is perhaps easier to put the epistemic tension in words with a non-presentist ontology. But temporal ontology is, in fact, a red herring, as my pessimistic induction shows. It is fine to have some inessential elements in one's argument for the sake of convenience. But we need to be careful to recognize them as they are. And it is particularly unfortunate for some A-theorists to think that the epistemic problem shows that presentism is the best version of A-theory --- that cannot be the moral of the story. Furthermore, my way of generalizing the problem, unlike Cameron's attempt, does not make the problem rely on a questionably stringent internalist conception of justification. The pessimistic induction works just as well under a more relaxed externalist standard for knowledge. So not only has the pessimistic induction shown that the epistemic problem is general, it has also shown the problem to be serious.

\section*{7 About Time?}

I mentioned that my argument is analogous to the pessimistic induction against scientific realism. One might then wonder whether the standard responses to the pessimistic induction against scientific realism apply directly to the pessimistic induction against A-theories. If the pessimistic induction against A-theories can be dealt with by a standard response to the pessimistic induction against scientific realism, one can't help suspect that the pessimistic induction against the A-theories isn't really a problem about time. If it isn't really about time, it is not a proper way to flesh out the epistemic problem against non-presentist A-theories after all. And I would have failed to offer a novel way to present the epistemic problem \emph{\textbf{about time}} to show that temporal ontology is irrelevant.

Standard defenses of scientific realism don't work for the A-theories. One typical response to the pessimistic induction against scientific realism is to bite part of the bullet. Some scientific realists argue that only some aspects of the previous scientific theories have been repeatedly shown to be wrong (e.g., the non-structural aspects). So they conclude that the pessimistic induction simply forces us to be more selective in our realism about scientific theories (e.g. Charkravartty (1998), Ladyman (1998)). Even if this is a feasible defense for a more modest version of scientific realism (e.g., Ladyman's structural realism), the approach does not apply to our case because the belief that this moment is present does not have multiple aspects like our scientific theories.

Other scientific realists respond by arguing that, although the immature scientific theories have been constantly replaced, once the scientific research on a topic has reached maturity, the theories are rarely replaced. So, it is argued that scientific realism should be understood as a realism about mature scientific theories (e.g. Psillos 1999: 105-108). This approach again does not apply to our case. There is no distinction between mature and immature belief about the location of presentness. When I believe that this moment is present, my belief is in no way more mature than my grandfather's belief.

So, whereas there are ways to be a scientific realist in the face of the pessimistic induction, those ways do not apply to A-theories. This is due to the simplicity of our belief about this moment's being present. Hence, although the pessimistic induction against A-theories and the pessimistic induction against scientific realism are similar in many ways, the epistemic problem as I articulate it is a unique problem for the A-theories, which arises if and only if we treat presentness as an objective feature of reality.

\section*{8 Is the Pessimistic Induction Admissible Evidence?}

There is another way one might try to solve the problem of pessimistic induction. One might say that the pessimistic induction I rely on, if made explicit, is something like the following:

\begin{quote}
\begin{description}
\item[Premise 1.] B\textsubscript{1} is false.
\item[Premise 2.] B\textsubscript{2} is false.
\item[Premise 3.] B\textsubscript{3} is false.
\item[\textellipsis]
\item[Conclusion.] B\textsubscript{0} is false.
\end{description}
\end{quote}

Now consider this question: Why do I think that my grandfather's belief that 1976 is the present is false? Because I believe that 2016 is present. Why do I think that, say, B\textsubscript{1} is false? Because I believe that this moment is present (i.e., that B\textsubscript{0} is true). So, the A-theorists might say, it seems that the only reason we have for accepting each of the premises in the inductive argument is that we \emph{\textbf{reject}} the conclusion of that same argument. If we can only reasonably accept the premises of an argument by rejecting its conclusion, we cannot reasonably accept that argument as our evidence for the conclusion.\footnote{Presumably, the same can be said about the optimistic induction.}

This is not a good response for two reasons. First of all, it is not obvious to me that the pessimistic induction needs premises as specific as ``B\textsubscript{1} is false". These premises are specific in the sense that each of them names a specific sample from the induction base (i.e., B\textsubscript{1}, B\textsubscript{2}, B\textsubscript{3},\textellipsis). According to the received view, it is part of the quantum entities' metaphysical nature that they (e.g., electrons) cannot be individually picked out and named.\footnote{E.g., see Lowe (1994); Pesic (2002); French \& Krause (2006); Huggett (2010).} But, presumably, inductive reasoning about quantum entities is legitimate. Hence, an inductive argument should not need premises that pick out specific samples from the induction base one by one. The pessimistic induction can do with a \emph{nonspecific} premise that says that the absolute majority of the beliefs about the location of the present before us are false. Accepting this nonspecific premise does not require the rejection or acceptance of B\textsubscript{0} beforehand.

Secondly, the response fails \emph{even if} I grant that <Premise 1, Premise 2, Premise 3,\textellipsis, Conclusion> is a proper way to flesh out the relevant pessimistic induction. Our A-theorist is right to point out that it is reasonable for us to accept those premises only if we reject the argument's conclusion (i.e., accepting that this moment is present). But that just means that \emph{our} epistemic access to presentness \emph{happens to be} such that it is reasonable for us to judge that a moment t is not present \emph{only if} we have already accepted that another moment is present. There is nothing \emph{in the concept} of an objective presentness that rules out the possibility that an epistemic agent can tell that t is not present even though she has no idea which moment is. So, our A-theorist's remark basically amounts to the following: although there is a pessimistic induction $\langle$Premise 1, Premise 2, Premise 3,\textellipsis, Conclusion$\rangle$ for the conclusion that this moment is not present, \emph{our peculiar epistemic predicament} does not allow us to endorse that induction as our evidence for the claim that this moment is not present. This doesn't help the A-theorists as much as they might hope.

Suppose there is a mathematical proof for the theorem M. But I am so mathematically incompetent that there is no way I can ever understand the proof. Nonetheless, I am told by the relevant experts that such a proof exists. Although my epistemic predicament prevents me from accepting the proof as my evidence for M (for I cannot accept something that I do not understand), being justified in believing that there is a proof out there gives me justification for accepting M. The fact that one has evidence for the presence of evidence for p (i.e., higher-order evidence) implies that one has evidence for p.\footnotemark

\footnotetext{This principle about higher-order evidence has been under much discussion. For defenders of the principle, see Feldman (2007), and Christensen (2010; 2013). For those who are against it, see Kelly (2005) and Fitelson (2012). Fitelson (2012) offered an alleged counter-example that challenges various interpretations of the principle. For the debate about Fitelson's counter-example and the subsequent attempts to formulate the principle in a more rigorous manner, see Feldman (2014), Roche (2014), and Tal \& Comesa�a (2015a; 2015b). My impression is that the consensus is now leaning towards accepting the evidential value of higher-order evidence.}

We have reason to think that there is a pessimistic inductive argument out there --- I have just written it out $\langle$Premise 1, Premise 2, Premise 3,\textellipsis, Conclusion$\rangle$! That is, we have evidence for thinking that there is a piece of evidence out there that supports Conclusion. That remains to be the case even though we happen to be in an epistemically unfortunate position that prevents us from taking this argument itself to be our evidence (analogous to the case about mathematical proof). Given that higher-order evidence entails evidence, even though our epistemic predicament prohibits us from endorsing the pessimistic induction itself as our evidence, we have evidence for believing that this moment is not present simply in virtue of having evidence for thinking that there is such an inductive argument out there. So, our A-theorist's remark does not help.

\section*{9 Conclusion}

It has been argued that the non-presentist A-theories' attempt to combine an objective presentness and a non-presentist ontology leads to an epistemic problem. I argue that the epistemic problem can be formulated in terms of a pessimistic induction, which renders temporal ontology irrelevant. I hence conclude that the epistemic problem is, in fact, a threat to the A-theoretic commitment to an objective presentness alone. And I've shown that the problem of the pessimistic induction is a hard problem by addressing two potential responses. Whereas Cameron tries to generalize the epistemic problem into a dialectically weak challenge for all A-theories in order to defend the non-presentist A-theories, I use the pessimistic induction to generalize the epistemic problem into a substantive threat to all A-theories. If we know that this moment is present, there is no objective presentness.

\nocite{*}
\bibliographystyle{plain}
\bibliography{Pessimistic_Induction_about_Objective_Presentness.bib}


\end{document}