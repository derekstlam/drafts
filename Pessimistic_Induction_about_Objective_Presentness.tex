\documentclass[a4paper,12pt]{article}
\usepackage{geometry}
%opening
\title{A Pessimistic Induction about the Objective Presentness}
\author{Derek Lam}

\begin{document}

\maketitle

\begin{abstract}
Some philosophers argue that non-presentist A-theories problematically implies that we do not know that this moment is present. The problem is generally presented as arising from the unholy combination of the A-theoretic ideology of a privileged presentness and a non-presentist ontology. The goal of this essay is to show that the epistemic problem can be rephrased in terms of a pessimistic induction. By doing so, I will show that the epistemic problem, in fact, stems from the A-theoretic ideology alone. Hence, once it is properly presented, the epistemic problem presents a serious threat to all A-theories.
\end{abstract}

\section*{1 The Epistemological Question}

Here's an observation: There were many people before us; they all believed that their moment, instead of ours, is the present (they didn't believe that their moment \emph{was} present). This observation has led philosophers to believe that there is an epistemic tension in non-presentist A-theories, namely, those theories that say that the tenses are real and \emph{at least} the past still exists.\footnotemark Suppose ``present" isn't a pure indexical like ``here" but denotes an objective feature of a special moment. The overwhelming majority in the history of mankind had beliefs about the location of presentness that are false (even though they \emph{were} true at that time). An epistemological question inevitably arises: if all those beliefs about the location of presentness before us are false, does that not give us a reason to think the same about our beliefs on the same topic? For some philosophers, this questions the plausibility of the attempts to combine an A-theoretic commitment to an objective presentness and a non-presentist temporal ontology (namely, the non-presentist A-theories).

\footnotetext{Theoretically, to be a non-presentist A-theorist, one just have to be an A-theorist who accepts the reality of the tenses and also think that some moments other than the present one exists. So, a non-presentist A-theorist can defend the view that only the future and the present moments exist and \emph{not the past}. I define non-presentist A-theory in the way I do here because the growing block view and the moving spotlight view are the only two non-presentist A-theories being defended and taken seriously in the literature. I want to focus on them and set aside other theoretical possibilities.}

The primary goal of this essay is to argue that, contrary to traditional wisdom, the aforementioned epistemic tension resides solely in the A-theoretic ideological commitment to an objectively privileged present. \footnotemark The tension has nothing to do with our temporal ontology. Hence it's a mistake to think that there is an epistemic problem only for the non-presentist A-theories. There is a serious epistemic problem for all A-theories.

\footnotetext{Here I am using Quine's (1951) famous distinction between a theory's \emph{ontological} and \emph{ideological} commitment. The ontological commitments of a theory are the objects it posits; the ideological commitments are the basic unanalyzable notions the theory introduces. The A-theories are committed to a \emph{primitive} notion ``present" that represents an objective presentness.}

Here's how I will proceed. In section 2, I'll introduce the epistemic problem for the non-presentist A-theories as it is standardly presented. In section 3, I'll argue that Cameron's (2015) recent attempt to generalize the problem for all A-theories fails. In section 4 - 6, I'll offer a new argument for the conclusion that the epistemic problem challenges all A-theories. To do that, I'll introduce a novel way to flesh out the epistemic problem that renders the ontology of time irrelevant. Finally, in section 7 - 8, I'll address two apparent solutions to the epistemic problem as I present it.

\section*{2 The Skeptical Problem}
It has long been pointed out that there is an epistemic tension in the non-presentist A-theories. The tension has been developed in different ways, but it has almost always been understood as a skeptical problem. That is, the problem is taken to be that, based on the fact about past people's beliefs, if one accepts non-presentist A-theory, then one has to accept that ``nobody ever knows which time is absolutely present." (Russell 2015: 3) --- in particular, we do not know that this moment is present.

The problem has been fleshed out in two ways. Very often, the skeptical problem is articulated in terms of the evidential indistinguishability of people before us and ourselves (Braddon-Mitchell 2004, Heathwood 2005, Merricks 2006). If the non-presentist A-theory is true, there are people before us who believe that, say, 1985 is the present year. Those beliefs are false. But those people are in a similar epistemic situation as I am from the first person's perspective as far as the location of the privileged presentness is concerned. So, from what I have evidential access to, I cannot tell the difference between those people before me and myself. As a result, I am not justified in believing that I am right and those people are wrong when I believe that this moment is present --- that is, I have no epistemic resources to rule out the possibility that I am one of those people who are stuck in the past and that this moment is in fact not present. Justification is necessary for knowledge. Thus, I do not know that this moment is present if non-presentist A-theory is true.

Others try to develop the epistemic tension in terms of the \emph{safety requirement} for knowledge instead (Cameron 2015; Russell 2015). The idea is that if a person knows that p, that person would still know that p at all close enough possible worlds. Given that all the past people still exist and are ``perfectly preserved" in the past, a world at which this moment is a tiny bit earlier than the objectively present moment with everything else remaining the same is a close enough possible world at which I believe but do not know that this moment is present (because that belief wouldn't be true). Therefore, if both the non-presentist A-theory and the safety principle for knowledge are true, we don't know that this moment is present.

Here is a fact: we certainly know that this moment is present. Therefore, either way, if the non-present A-theories entail that we do not know that, there is a problem with the views. This is the standard understanding of the epistemic problem for non-presentist A-theories.

Contrary to the standard understanding, it is worth mentioning that there are philosophers who think that the epistemic tension is in fact not about our knowledge of the location of the privileged present. Miller (2016: 320-321), for example, argues that the \emph{real} problem is that the non-presentist A-theorists cannot account for how we get to have the \emph{justified} belief that this moment is the privileged present given their metaphysic.

I disagree. Surely we can formulate the epistemic problem in the following way: if some form of non-presentist A-theory is right, we have no justification for believing that this moment is present. That would be a problem for the non-presentist A-theories. But \emph{why} is it a problem? The answer is that we clearly have knowledge that this moment is present \emph{and} knowledge requires justification. So, I think it is a mistake to think that the real problem is not about knowledge because it is about justification. I suppose Miller wouldn't be satisfied if the non-presentist A-theorists answer her concern by offering an account of how we come to be justified in believing that this moment is present while admitting that that justified belief does not constitute knowledge.

Furthermore, Russell's (2015) formulation of the epistemic problem is about the safety requirement \emph{for knowledge}. Hence, the challenge is directly about the non-presentist A-theorists' alleged inability to accommodate our knowledge of this moment's being present. If one accepts some form of reliabilism about justification, then, by violating the safety requirement for knowledge, one's belief could also lack reliability to count as being justified. But the problem as Russell formulates it does not have to (and hence should not) be held hostage to a substantive theory of justification, namely reliabilism. As long as one accepts that there is a safety requirement for knowledge, there is a problem for the non-presentist A-theories --- whether or not we also accept reliabilism and think that there is also a problem about justification. So, not only is it not obvious to me that the problem is about justification instead of knowledge, I also find it debatable that the problem is primarily about justification.

Due to these considerations, I will stick to the standard understanding that the epistemic tension is a skeptical problem about knowledge. It is a problem that says, if some version of non-presentist A-theory is true, we do not have knowledge that this moment is present; and we clearly know that this moment is present. Hence, we have a \emph{reductio}.

\section*{3 Skeptical Problem for \emph{Presentism}}

A non-presentist A-theory has an ideological component and an ontological component. The ideological component is about accepting a privileged presentness as an objective feature of reality. The ontological component is the claim that not only the present moment but at least the past moments also exist. As we have seen, the traditional presentations of the skeptical problem capitalize on the idea that this particular blend of ideology and ontology implies that people before us exist and have very different beliefs about the location of the privileged presentness, despite experiencing the world in the same way as we do. There is trouble either because those people have the same kind of evidential access as we have, or because we could very easily be one of them. Presented as such, the problem is meant to target the non-presentist A-theorists specifically.

In his recent book, Cameron (2015) challenges this orthodoxy. He argues that the skeptical problem is either not a problem for the non-presentist A-theories, or it is at best a dialectically weak problem for all A-theories, presentism and non-presentism alike. Here is basically his reasoning:

\begin{quote}
\begin{description}
\item[{[1]}] The epistemic problem is to be understood either in terms of evidential indistinguishability or the safety requirement for knowledge.
\item[{[2]}] If the epistemic problem is understood in terms of evidential indistinguishability, it requires a very stringent internalist conception of justification that would make the epistemic problem applicable to presentism as well. (ibid: 24)
\item[{[3]}] If the epistemic problem is presented with a very stringent internalist conception of justification, its challenge would be dialectically weak.
\item[{[4]}] The safety requirement for knowledge does not create an epistemic problem for non-presentist A-theories at all, according to the right semantics for tensed counterfactuals. (ibid: 40)
\item[{[C]}] The epistemic problem is either a dialectically weak problem for all A-theories or it is not a problem (for non-presentist A-theories) at all.
\end{description}
\end{quote}

On the one hand, \emph{if} Cameron is right about the semantics for tensed counterfactuals (according to [4]), the epistemic problem cannot be fleshed out in terms of safety. On the other hand, \emph{if} he is not right, we can formulate an epistemic problem for the non-presentist A-theories that does not apply to presentism. Since Cameron's goal is to show that there is no serious problem for the non-presentist A-theories, he needs [4] to be true. On the contrary, my goal is to argue that the epistemic problem poses a genuine threat to both presentist and non-presentist A-theories --- I am not trying to defend non-presentist A-theories. And my main beef about the argument [1] - [C] is not [4], but [1] and [2]. So whether [4] is true or not is insignificant for my purpose. Even if Cameron turns out to be wrong about [4], that only means the epistemic problem can be presented narrowly to target non-presentist A-theories specifically. That does not contradict my contention that the epistemic problem is in fact a general problem for all A-theories. I have nothing interesting to say about the semantics of tensed counterfactuals. Therefore, I will assume Cameron is right about [4] and set the epistemic problem formulated in terms of the safety principle aside in the rest of this essay.

I think [2] is false. Cameron believes that [2] is true because of the similarity he sees between the epistemic problem and the problem of external world skepticism. In the case of external world skepticism, the issue is that there is no evidence that we have access to that allows us to rule out the \emph{\textbf{epistemic possibility}} that we are BIVs. Cameron thinks that we should understand the epistemic problem against non-presentist A-theories in the same way: if some form of non-presentist A-theory is true, we do not have evidential resources to rule out the \emph{\textbf{epistemic possibility}} that we are stuck in the past just like the people exist before us. Since the problem is about epistemic possibility, the fact that presentism is true or even metaphysically necessary would not help, for that metaphysical thesis alone would not rule out unwanted epistemic possibilities (ibid: 25).

Let me begin by pointing out that, if Cameron was right that the epistemic problem is about our lack of evidential resources to rule out the epistemic possibility that we are stuck in the past, the problem wouldn't threaten only the A-theories. It threatens the B-theories too. Suppose we are to understand the epistemic problem in the mold of external world skepticism: just as the presentists cannot appeal to the alleged truth of presentism because that presentism is false is an epistemic possibility left open by the mere truth of presentism, the B-theorists also cannot appeal to the B-theoretic idea that being present is not an objective feature of reality in order to rule out the epistemic possibility that we are stuck in the objective past given that we are yet to rule out the epistemic possibility that the B-theory is wrong about that. What this shows, I fear, is that understanding the epistemic problem in the mould of external world skepticism fails to capture what the epistemic problem is really getting at and mischaracterizes the challenge it posts.

More importantly, when the epistemic problem is presented in terms of evidential indistinguishability, it says that, assuming some form of non-presentist A-theory, there \emph{are} people who have the same sort of evidential access as we have and believe that this moment of ours is not present; and that should force us to be skeptical of our belief that this moment is present.

Consider the following analogous scenario. Suppose a group of people have dinner together. When the check arrives, we all take a quick look at it. I think each of us has to pay \$22. But it turns out that everyone else thinks that it is \$26 per person. Presumably, given that there are so many \emph{actual} people who come to a different conclusion based on the same evidence, I should not claim that I know that each of us has to pay \$22.

If we were to apply what Cameron says about the epistemic problem to the dinner case, basically he would be saying that we should understand my skeptical reasoning in the situation in the mold of external world skepticism: surely I should not claim that I know that each of us has to pay \$22; but whether there actually are people who think that it is \$26 per person is not the point, what is crucial is merely the fact that it is epistemically possible that there is someone at the table who looked at the check and concluded that it is \$26 per person.

I think this is clearly not the point of the original reasoning in the dinner case. Surely if I were a radical skeptic who endorsed some extremely demanding criterion of knowledge, I would have some radically skeptical reason to deny knowing that the meal is \$22 per person even if no one actually thinks otherwise. But we don't need to be radically skeptical to find the self-doubt in the dinner case intelligible. In the original reasoning, the fact that people who \emph{actually} think that the meal costs \$26 per person exist is crucial to the conclusion that I should not claim I know that it is \$22 per person. I \emph{wouldn't} have drawn the same skeptical conclusion if I hadn't thought that some people \emph{actually} think that the meal costs \$26 per person, even if it was \emph{epistemically possible} that they think so secretly. Thus, it would be a distortion of the original reasoning to insist that what is really going on in the dinner case is only about the mere epistemic possibility that someone might think that the meal costs \$26 per person.

My view on Cameron's [2] is the same. Assuming that some form of non-presentist A-theory is true, there are many people who believe that this moment is not present base on the same kind of evidence as we do. When philosophers argue that that undermines our claim to know that this moment is present, the argument is the same in spirit as the one in the dinner case. Certainly one can reach the same skeptical conclusion on a radically skeptical basis (i.e., that the epistemic possibility of otherwise alone is sufficient for skepticism). But there is no basis for insisting that the epistemic problem has to be understood in the mold of radical external world skepticism --- doing so is quite uncharitable especially when we agree with Cameron that the epistemic problem is dialectically weak presented this way.\footnotemark

\footnotetext{That is, given [3], interpreting the epistemic problem in the mould of external world skepticism is rather uncharitable even if one remains unconvinced that interpreting the problem this way threatens the B-theory as well. I believe the dinner example should be able to show that.}

\emph{If} we choose to understand the skeptical challenge to non-presentist A-theories as a radically skeptical one, then it is indeed a challenge to all A-theories. Cameron is right about that. And the challenge would indeed be a dialectically weak one given that there is no strong reason to accept such a demanding criterion of knowledge that requires ruling out all alternative epistemic possibilities based on the evidence one has access to. But Cameron fails to offer a convincing case for [2], which requires us to read the epistemic problem uncharitably in the mold of the argument for external world skepticism.

\section*{4 A Pessimistic Induction}

I agree with Cameron's conclusion to a certain extent: the epistemic problem is a problem for all A-theories (but not the B-theory). However, I think his argument for the conclusion is flawed, as I have argued. In the rest of this paper, I will offer a new argument for the conclusion (one that involves rejecting [1] as well). But whereas he aims to defend non-presentist A-theories by showing that, if there is an epistemic problem, it is at best a weak problem for all A-theories (due to [2] and [3]), I aim to raise a genuine epistemic challenge to all forms of A-theory.

Here is my strategy. In the rest of this section, I will offer a novel way to flesh out the epistemic problem. It does not appeal to evidential indistinguishability, nor does it rely on the safety principle (hence rejecting [1]). In section 5 and 6, I will show that the way I flesh out the epistemic problem renders the ontology of time irrelevant. Hence, under my way of presenting the epistemic problem, it is a threat to all A-theories, presentist and non-presentist alike. I will then argue that, given that there is a way to flesh out the epistemic problem independent of temporal ontology, trying to solve the epistemic problem by appealing to a presentist ontology is trying to solve the problem in the wrong way: it is to address a problem by targeting something contingent on a particular presentation of a problem instead of dealing with the problem itself.

People before us held beliefs about the location of presentness, e.g., my grandfather once believed that 1967 is the present year. Let's name those beliefs B\textsubscript{1}, B\textsubscript{2}, B\textsubscript{3},\textellipsis. I take it that it's unproblematic to name things before us, regardless of one's ontology of time. After all, we should be able to name Napoleon (I just did!) no matter what we think about the ontology of time. Suppose the non-presentist A-theorists are right that the presentness is an objective property. And there's only one privileged present. So B\textsubscript{1}, B\textsubscript{2}, B\textsubscript{3},\textellipsis \emph{\textbf{are}} almost all false; e.g., my grandfather's belief that 1967 is the present year is not true. Since the beliefs B\textsubscript{1}, B\textsubscript{2}, B\textsubscript{3},\textellipsis are not randomly put together (they are beliefs on the same topic), \emph{by induction}, we have reason to think that \emph{my belief} on the same subject matter, namely, my belief that this moment is present (let's name this belief B\textsubscript{0}), is also false. Notice that this is analogous to the pessimistic induction against scientific realism: when we look at the history of science, most of the theories that were once accepted turn out to be false; so, by induction, it seems that we should think that our current theories are also false. Let me call this inductive reasoning about the truth-value of B\textsubscript{0} the \textbf{pessimistic induction}.

Whereas B\textsubscript{1}, B\textsubscript{2}, B\textsubscript{3},\textellipsis \emph{are} almost all false, they \emph{used to be} true. My grandfather's belief that 1967 is present \emph{is false} but it \emph{was true} when it was 1967. So, there is another induction in the neighborhood. The past beliefs B1, B2, B3,\textellipsis \emph{were} all true. Hence, by a \emph{different} induction, my belief B\textsubscript{0} is true. Past beliefs about the location of presentness were true (even though they are false now), so we have reason to believe that our beliefs about the location of presentness are true. So this moment is present. Let me call this the \textbf{optimistic induction}.

The pessimistic induction is based on how most of those beliefs \{B\textsubscript{1}, B\textsubscript{2}, B\textsubscript{3},\textellipsis\} \emph{are}. And the optimistic induction is about how most of those beliefs \emph{were}. When we want to figure out how things are, an induction based on how things are simpliciter overrides an induction based on how things were. Here is an example to illustrate why this claim is true. I've eaten oranges my entire life. Oranges used to be sweet. But things have changed. Oranges these days are sour. I am wondering whether the slice of orange I am about to put in my mouth is sweet or sour. There are two inductions to be had here. Based on the fact that oranges \emph{have always been} sweet, I have an inductive reason to believe that this orange is sweet. This is an optimistic induction. But knowing that oranges \emph{are} all sour, I also have an inductive reason for believing that this orange is sour. This is a pessimistic induction. When I am interested in finding out whether this orange is sweet or sour, it seems obvious that the pessimistic induction based on how oranges are overrides the optimistic induction based on how oranges were. Similarly, when we are interested in finding out whether my belief B\textsubscript{0} is true, the pessimistic induction overrides the optimistic induction and the fact that almost all of B\textsubscript{1}, B\textsubscript{2}, B\textsubscript{3},\textellipsis used to be true would not help undermine the pessimistic induction which concludes that my belief that this moment is present is false, i.e., that this moment is not present.

The pessimistic induction I have just introduced sounds similar to the reasoning based on a principle Cameron calls \emph{Statistical Knowledge}: ``If you believe p on basis E then, if most of the people who believe p on basis E believe falsely, you do not know p on basis E". (2015: 43) The crucial difference, however, lies in this. For Cameron, the principle Statistical Knowledge is just another way to capture the safety requirement for knowledge: ``I think [Statistical Knowledge] only sounds appealing because [the safety requirement] is appealing". (ibid: 43) On the contrary, the rational pull of my pessimistic induction does not rely on the modal character of knowledge, e.g., the safety requirement. All I need for the pessimistic induction about the objective presentness to work is the innocent idea of the legitimacy of induction about the objective features of things. The rational appeal of induction does not stem from the safety principle; even a modal skeptic can find induction reasonable.

Let me be clear: I do \textbf{not} endorse the pessimistic induction. Obviously, we know that this moment is present. Obviously, to doubt that this moment is present based on the pessimistic induction is absurd. As a result, we have the epistemic problem in the form of a \emph{reductio} argument. If presentness were an objective feature of some special moment, i.e., if the A-theory were true, it would be an objective fact of the matter that B\textsubscript{1}, B\textsubscript{2}, B\textsubscript{3},\textellipsis are all false. Since drawing conclusions about objective features of things by induction is a legitimate way of reasoning, the A-theory would leave us no choice but to accept that the pessimistic induction is legitimate and, as a consequence, that we don't know that this moment is present. This consequence is absurd. By reductio, we should conclude that presentness is not an objective feature of reality, i.e., the A-theory is false.

The force of the reductio can be made even more pronounced by noting that, if the B-theory were true instead, B\textsubscript{1}, B\textsubscript{2}, B\textsubscript{3},\textellipsis are all true because ``present" would be a pure indexical. So, there won't be a legitimate pessimistic induction by the light of the B-theory. It is only when ``present" is taken to denote an objectively privileged moment as the A-theories have us believe, B\textsubscript{1}, B\textsubscript{2}, B\textsubscript{3},\textellipsis are all false.\footnotemark As long as that is the case, we have no choice but to embrace the pessimistic induction. Unlike the B-theory, the A-theory lacks the theoretical resources to explain in a principled manner why the pessimistic induction is no good.

\footnotetext{Or \emph{\textbf{almost}} all false, if we do not wish to assume that this moment is present and that none of the moments before us is present.}

\end{document}